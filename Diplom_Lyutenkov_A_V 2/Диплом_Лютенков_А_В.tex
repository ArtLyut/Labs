\documentclass[12pt, a4paper]{extarticle} 
\usepackage{amsfonts} 
\usepackage[T2A]{fontenc} 
\usepackage[utf8]{inputenc} 
%\usepackage{mathtext} 
\usepackage{amsmath, amsfonts, amssymb} 
\usepackage[russian]{babel} 
\usepackage[body={17.5cm, 23.5cm},left=3cm, top=2cm, right=2cm]{geometry} 
\usepackage{graphicx} 
\usepackage{float}
\usepackage{amsthm}
\usepackage{blindtext} 
\usepackage{fancyhdr} 
\usepackage{graphicx} 
\usepackage{ragged2e} 
\usepackage{color}
\usepackage{moreverb}
\usepackage[noend]{algorithmic}
\usepackage{listings}
\usepackage{misccorr} % в заголовках появляется точка, но при ссылке на них ее нет 
\usepackage{indentfirst} % после заголовков ставится абзацный отступ 
%\parindent{1.25cm} 
\graphicspath{images/} 
\setcounter{tocdepth}{6} 
\newcommand{\eps}{\varepsilon} 
\newcommand{\re}{\operatorname{Re}} 
\newcommand{\im}{\operatorname{Im}} 
\renewcommand{\labelitemi}{$-$} 
\renewenvironment{itemize}[1][{---\hfil}]{\begin{list}{#1}{\topsep=0pt\parsep=0pt plus 1pt\itemsep=\parsep\leftmargin=0pt \itemindent=\parindent}\addtolength{\itemindent}{\labelwidth}}{\end{list}}
\newtheorem{stat}{Утверждение} 
\newtheorem{theorem}{Теорема}
\newtheorem*{lemma}{Лемма} 
%\setcounter{page}{} 
\begin{document} 
\thispagestyle{empty} 
\medskip 
 
\begin{center} 
\textbf{МИНОБРНАУКИ РОССИИ\\ 
\vspace{0.5cm} 
Федеральное государственное бюджетное образовательное\\ 
учреждение высшего образования\\ 
«Ярославский государственный университет им. П.Г. Демидова»}\\ 
\vspace{0.5cm} 
{Кафедра математического анализа}\\ 
\vspace{1.5cm} 

\end{center}
\begin{flushright} 
	Сдано на кафедру\\
		« 
	\underline{\phantom{aaa}} 
	» 
	\underline{\phantom{aaaaaaaaaaaaa}} 2018 г.\\ 
	Заведующий кафедрой\\
	\underline{\phantom{aaa}д. ф.-м. н., доцент\phantom{aaa}}\\ 
	\vspace{0.1cm} 
	\underline{\phantom{aaaaaaaaaaaaa}} М.В. Невский
\end{flushright}
\vspace{3cm} 
\begin{center} 
Выпускная квалификационная работа\\ 
\vspace{0.5cm} 
\textbf{Нормы интерполяционных проекторв и экстремальные симплексы}\\ 
(Направление подготовки бакалавров 01.03.02 Прикладная математика и информатика)
\vspace{3cm} 
\end{center} 
 
\begin{flushright} 
Научный руководитель\\ 
	\underline{\phantom{aaa}канд. ф-м. н., доцент\phantom{aaa}}\\ 
	\vspace{0.1cm} 
\underline{\phantom{aaaaaaaaaaaaa}} А.Ю. Ухалов\\ 
« 
\underline{\phantom{aaa}} 
» 
\underline{\phantom{aaaaaaaaaaaaa}} 2018 г.\\ 
\vspace{0.5cm} 
Студент группы \underline{\phantom{a}ПМИ-41БО\phantom{a}}\\ 
\vspace{0.1cm} 
\underline{\phantom{aaaaaaaaaaaaa}} А. В. Лютенков\\ 
« 
\underline{\phantom{aaa}} 
» 
\underline{\phantom{aaaaaaaaaaaaaa}}2018 г.\\ 
\vspace{1cm} 
\end{flushright} 
\begin{center} 
Ярославль 2018 г.
\vspace{-1cm}  
\end{center} 

 
\justify 
\setlength{\parindent}{1.25cm} 
\newpage 
\thispagestyle{empty} 
\setcounter{page}{2} 
\begin{center}
Реферат
\end{center}

Объем 23 с., 3 гл., 2 табл., 10 источников, 2 прил.\newline
Ключевые слова: {\bf ныевырожденный симплекс, интерполяционный проектор, норма проектора, минимальная норма проектора, минимизация функции, Dlib, Eigen, Boost, алгоритм Бройдена-Флетчера-Голдфарба-Шанно}

Объектом исследования является задача о построении минимального проектора (проектора, имеющего минимальную норму) при интерполяции непрерывной на кубе функции с помощью полиномов n переменных степени не выше единицы. Оценки для норм минимальных проекторов играют важную роль в теории приближений.  Точные значения минимальной нормы проектора в настоящее время известны только для n=1,2,3,7.

Цель работы ---разработать компьютерную программу для численной минимизации функции многих переменных и применить ее для решения задачи о минимальном проекторе.

В результате была реализована компьютерная программа для получения верхних оценок минимальной нормы проектора. Удалось улучшить оценки норм минимальных проекторов в случаях n=5,6,10,14,18,20.

\newpage 
%\thispagestyle{empty} 
\tableofcontents 
\newpage 
\section*{Введение} 
\addcontentsline{toc}{section}{Введение} 

В данной ВКР рассматривается задача о построении минимального проектора (проектора, имеющего минимальную норму) при интерполяции непрерывной на кубе функции с помощью полиномов n переменных степени не выше единицы. Оценки для норм минимальных проекторов играют важную роль в теории приближений. Неравенство Лебега связывает норму проектора с величиной наилучшего приближения функции многочленами соответствующей степени. Этим, в частности, и обусловлен интерес к изучению минимальных проекторов и к получению оценок их норм. Точные значения минимальной нормы проектора в настоящее время известны только для n=1,2,3,7.

Также описывается реализованая в рамках данной работы компьютерная программа, которая решает задачу численной оптимизации функции многих переменных для нахождения верхних оценок минимальной нормы проектора. С помощью реализованной программы удалось улучшить верхние оценки минимальных норм проекторов при n=5,6,10,14,18,20.



\newpage 
\newpage
\section{Задача линейной интерполяции на n-мерном кубе}\label{s1}
Положим $Q_ n:= [0..1]^n$, где $n \in \mathbb{R}^n$, $Q_ n$ --- n-мерный куб, множество вершин куба будем обозначать как $ver(Q_n)$.  $\Pi_1(\mathbb{R}^n)$ --- совокупность многочленов n переменных степени $\leqslant 1$. Пусть S --- невырожденный сиплекс в $\mathbb{R}^n$, вершины симплекса зададаются, как $x^{(j)} = (x_1^{(j)},...,x_n^{(j)}),  j = 1,\ldots,n $. Рассмотрим матрицу A:
$$A := {\begin{pmatrix}
	x_1^{(1)}& \dots & x_n^{(1)}& 1\\
	\vdots & \ddots & \vdots & \vdots \\
	x_1^{(n+1)}& \dots & x_n^{(n+1)}&  1
	\end{pmatrix}}. \eqno (1.1)$$
Скажем, что набор точек $x^{(j)}$ --- допустим для интерполяции многочленами из $\Pi_1(\mathbb{R}^n)$. Это условие эквивалентно тому, что
матрица A является невырожденной.  
\newline
$\Delta := det(A)$, определитель, который получается  из  $\Delta$ заменой j-й строки на строку $(x_1, \dots, x_n, 1)$. Многочленый $\lambda_j(x) := \Delta_j(x)/\Delta$ из $\Pi_1(\mathbb{R}^n)$ называются базисными многочленами Лагранжа симплекса S и обладают свойством  $\lambda_j(x^{k}) = \delta^k_j $, где $\delta^k_j $ --- символ Кронакера. $\lambda_j = l_{1j}x_1 + \dots + l_{nj}x_n + l_{n+1j}$, коэффициентны $l_{ij}$ составляют столбцы матрицы $$A^{-1} = {\begin{pmatrix}
	\dots & l_{1,j}&\dots\\
	\vdots & \vdots & \vdots \\
	\dots& l_{n,j} & \dots\\
	\dots& l_{n+1,j} & \dots\\
	\end{pmatrix}}.\eqno (1.2)$$
\newline
Так как $\lambda_j(x^{k}) = \delta^k_j $ любой многочлен $p \in \Pi_1(\mathbb{R}^n)$ удовлетворяет равенству 
$$p(x) = \sum\limits_{j = 1}^{n+1} p(x^{(j)})\lambda_j(x). \eqno (1.3)$$
\newline
Так как $det(A) \neq 0 $, то для любой  $f \in C(Q_n)$, где $C(Q_n)$ --- совокупность $f : Q_n \rightarrow \mathbb{R}$ найдется единственный многочлен $p \in \Pi_1(\mathbb{R}^n	)$ удовлетворяющий условиям:
$$p(x^{(j)}) = f(x^{(j)}). \eqno (1.4)$$
\subsection{Интерполяционный проектор}
Введем в рассмотрение оператор $P : C(Q_n)  \rightarrow \Pi_1(\mathbb{R}^n)$, который далее будем называть интеполяционным проектором. Интерполяционный проектор по системе узлов $x^{(j)}$ определяется с помощью равенств:
$$Pf(x^{(j)}) = f_j := f(x^{(j)}),  j = 1,\dots, n+1. \eqno (1.1.1)$$

Из этих равенств следует, что данный оператор является линейным и справедлив следующий аналог интерполяционной формулы Лагранжа:
$$Pf(x^{(j)}) = p(x) = \sum\limits_{j=1}^{n+1} f_j\lambda_j(x). \eqno(1.1.2)$$

\subsection{Норма интерполяционного проектора. Минимальная норма проектора }
Обозначим $||P||$ норму оператора Р. Эта величина зависит от от узлов $x^{(j)}$. 

\begin{lemma}
Для любого интерполяционного проектора $P : C(Q_n)\rightarrow \Pi_1(\mathbb{R}^n)$ и симплекса $S$ с вершинами в его узлах имеет место равенство 
$$||P|| = \max\limits_{x \in ver(Q_n)} \sum\limits_{j = 1}^{n+1} |\lambda_j(x)|\eqno (1.2.1)$$
\end{lemma}
Доказательство этого утверждения можно найти в монографии \cite{1}.
\newline
Обозначим через $\theta_n$ минимальну норму проектора, при условии, что все узлы принадлежат кубу $Q_n$:
$$\theta_n := \min\limits_{x^{(j)} \in Q_n}||P||\eqno (1.2.2)$$
Интерполяционный проектор $P^*$ c нормой $||P^*|| = \theta_n$ назовем минимальным. 

Главной задачей настоящей работы является уточнение оценок для миниммальной нормы проектора при некоторых значениях $n$. 

Отметим, что минимальная норма проектора достигается на границе куба $Q_n$, т.е. в том случае, когда все вершины невырожденного симплекса принадлежат границе $Q_n$, доказательство данного факта приводится в книге М.В. Невского (\cite{1})

В монографии \cite{1} приводятся следующие общие оценки:
$$\frac{1}{4}\sqrt{n}< \theta_n <3\sqrt{n},$$
$$3-\frac{4}{n+1}\leqslant\theta_n.$$
Приведем некоторые оценки для $\theta_n$:\newline
\begin{flushright}
	Таблица 1.2.1
\end{flushright}
%\begin{table}[H]
\begin{center}
	{\bf Примеры оценок $\theta_n$}\\
	\begin{tabular}{|c|c|} \hline
		n&$\theta_n$\\ \hline
		1 & $\theta_n = 1$\\ \hline
		2 & $\theta_n = 1.89\ldots$\\ \hline
		3 & $\theta_n = 2$\\ \hline
		4 & $2.2\ldots \leqslant\theta_n\leqslant2.33\ldots$\\ \hline
	    5 & $2.33\ldots \leqslant\theta_n\leqslant2.6\ldots$\\ \hline
		6 & $2.42\ldots \leqslant\theta_n\leqslant3$\\ \hline
		7 & $\theta_n = 2.5$\\ \hline
	\end{tabular}
\end{center}
%\end{table}
Более подробные cведения о свойствах констант  $\theta_n$ и их оценках приводятся в книге М.~В.~Невского \cite{1}.
\newpage
\section{Компьютерная программа для расчета $\theta_n$} 

В рамках данной ВКР была реализована компьютерна программа для уточнения оценок $\theta_n$. Задача об отыскании минимальной нормы проектора сводится к задаче отыскания минимума функции многих переменных. 
Норма проектора вычисляется по формуле(1.2.1), зная это, зададим целевую функцию для минимизации. 
$$F(A) = \max\limits_{x \in ver(Q_n)} \sum\limits_{j = 1}^{n+1} |\lambda_j(x)|\eqno (2.1.1)$$
Где А --- марица, которая имеет вид (1.1).

Так как задача минимизации функции (2.1.1) является трудоемкой, целесообразным является применение сторонних программных библиотек, поставляющих решения задач линейной алгебры, численных методов и медодов оптимизации, существующих на рынке свободно распространяемого програмного обеспечения.

Программа реализована на языке программировани C++ с использованием библиотек Dlib, Boost, Eigen, которые предоставляют необходимый функционал для оптимального решения поставленной задачи.
\subsection{Dlib}

Dlib - это универсальная кроссплатформенная программная библиотека, написанная на языке программирования C++. На ее дизайн в значительной степени влияют идеи проектирования по контракту и разработки программного обеспечения на основе компонентов. Таким образом, это, прежде всего, набор независимых программных компонентов. Это программное обеспечение с открытым исходным кодом, выпущенное под лицензией Software Boost.

С момента начала разработки в 2002 году, Dlib вырос до широкого спектра инструментов. По состоянию на 2016 год он содержит программные компоненты для работы с сетями, потоками, графическими пользовательскими интерфейсами, структурами данных, линейной алгеброй, машинным обучением, обработкой изображений, интеллектуальным анализом данных, анализом XML и текста, численной оптимизацией, байесовскими сетями и многими другими задачами. В последние годы большая часть развития была сосредоточена на создании широкого набора инструментов статистического машинного обучения, в 2009 году описание этого набора инструментов Dlib было опубликовано в журнале «Journal of Machine Learning Research» \cite{2}. С тех пор он используется в широком диапазоне областей.

\subsection{Eigen}
Eigen - это высокоуровневая библиотека C++  для линейной алгебры, матричных и векторных операций, геометрических преобразований, численных вычислений и связанных с ними алгоритмов. Eigen является библиотекой с открытым исходным кодом, лицензированной в MPL2, начиная с версии 3.1.1. Более ранние версии были лицензированы под LGPL3 + \cite{3}.

Eigen реализуется с использованием метода метапрограммирования шаблонов выражений, то есть она строит деревья выражений во время компиляции и генерирует собственный код для их вычисления. Используя шаблоны выражений и модель затрат операций с плавающей запятой, библиотека выполняет свою собственную развертку и векторизацию цикла \cite{4}.

\subsection{Boost}
Boost представляет собой набор библиотек для языка программирования C++, который обеспечивает поддержку задач и структур, таких как линейная алгебра, генерация псевдослучайных чисел, многопоточность, обработка изображений, регулярные выражения и модульное тестирование. Он содержит более восьмидесяти отдельных библиотек.

Большинство библиотек Boost лицензированы по лицензии Boost Software, которая позволяет использовать Boost как с бесплатными, так и с проприетарными проектами программного обеспечения. Многие основатели Boost находятся в комитете по стандартам C++, и несколько библиотек Boost были приняты для включения в стандарт C++ Technical Report 1 и C++ 11 \cite{5}.

\subsection{Реализация программы}
Программа реализована в виде консольного приложения. Приложение зависит от выше указанных программных библиотек, также использует страндартную библиотеку шаблонов STL языка C++, листинг основной части кода программы приводится в Приложении~1. 

Так как для вычисления нормы проектора необходим перебор по вершинам куба, то асимптотика алгоритма отыскания $\theta_n$ будет порядка $O(2^n)$, что требует оптимального выполнения некоторых операций, для обеспечения наилучшего времени работы программы. Для оптимальной работы с матрицами и операций над ними (напимер вычисление обратной матрицы) используется библиотека Eigen.

Для оптимизации целевой функции (2.1.1) используются решения, поставляемые библиотекой Dlib. Dlib предоставляет метод минимизации нелинейной функции многих переменных внутри куба $Q_n$, с возможностью использования различных стратегий поиска минимума функции. Для решения текущей задачи стратегиями поиска были выбраны алгоритмы: BFGS и L-BFGS.
\subsubsection{Алгоритм Бройдена-Флетчера-Голдфарба-Шанно (BFGS)}
При численной оптимизации алгоритм Бройдена-Флетчера-Голдфарба-Шанно (BFGS) является итерационным методом решения неограниченных задач нелинейной оптимизации \cite{6}.

Метод BFGS относится к квази-ньютоновским методам, классу методов оптимизации восходящего подъема, которые ищут стационарную точку (предпочтительно дважды непрерывно дифференцируемой) функции. Для таких задач необходимым условием оптимальности является то, что градиент равен нулю. Метод Ньютона и методы BFGS не гарантируют сходимости, если функция не имеет квадратичного разложения Тейлора вблизи оптимума. Тем не менее, BFGS доказал свою хорошую производительность даже для негладкой оптимизации \cite{7}.

В квази-ньютоновских методах матрицу гессиана вторых производных не нужно оценивать напрямую. Вместо этого матрица гессиана аппроксимируется с использованием обновлений, определяемых оценками градиента (или приблизительными оценками градиента). Квази-ньютоновские методы являются обобщениями метода секущих для нахождения корня первой производной для многомерных задач. В многомерных задачах уравнение секущей не определяет уникальное решение, а квази-ньютоновские методы отличаются тем, как они ограничивают решение. Метод BFGS является одним из самых популярных членов этого класса \cite{8}. Также широко используется L-BFGS, который представляет собой версию BFGS с ограниченной памятью, которая особенно подходит для задач с очень большим количеством переменных (например,> 1000). Вариант BFGS-B \cite{9} обрабатывает простые ограничения например кубом.

Алгоритм назван в честь Чарльза Джорджа Бройдена, Роджера Флетчера, Дональда Голдфарба и Дэвида Шанно.
%\begin{lstlisting}

Схема алгоритма:
\begin{flushleft}
дано $ \epsilon , x_{0} $ \newline
инициализировать $C_{0}$ \newline
$ k=0$ \newline
while $||\nabla f_{k}||>\epsilon$ \newline
найти направление $p_{k}=-C_{k}\nabla f_{k}$ \newline
вычислить $x_{k+1}=x_{k}+\alpha _{k}p_{k} \alpha _{k} удовлетворяет условиям Вольфе $ \newline
обозначить $s_{k}=x_{k+1}-x_{k} и y_{k}=\nabla f_{k+1}-\nabla f_{k} $ \newline
вычислить $C_{k+1}$ \newline
$ k=k+1$ \newline
end \newline
\end{flushleft}
%\end{lstlisting}


\section{Выбор начальных приближений для численной минимизации} 

Норма проектора $\|P\|$ в $R^n$, которую требуется минимизировать в данной работе --- функция $n(n+1)$ переменных. При $n<7$ выбор начального приближения для работы алгоритма 
удавалось осуществлять с помощью некоторого числа случайных попыток.
При $n>7$ найти случайным образом хорошее начальное приближение оказалось затруднительным.

По этой причине, при $n>7$ в качестве начальных приближений использовались узлы проекторов
на которых были получены оценки, приведенные в монографии М. В. Невского~\cite{10} ---
вершины симплексов максимального объема в $Q_n$.

Симплексом максимального объема в кубе $Q_n$ называется 
такой $n$-мерный симплекс $S\in Q_n$, что для любого $n$-мерного симплекса $S' \in Q_n$ 
выполняется неравенство $\operatorname{vol}(S)\geq \operatorname{vol}(S')$.
М. В. Невским доказано следующее утверждение (см~\cite{1}).

{\it Если $S$ --- симплекс максимального объема в $Q_n$, $P$ --- интерполяционный 
проектор с узлами в вершинах $S$, то
$$
\|P\|\asymp\theta_n.
$$ 
}
Здесь $f(n)\asymp g(n)$ означает, что существуют такие константы $c_1,c_2>0,$
не зависящие от $n$, что выполняется 
$c_1 g(n)\leq f(n)\leq c_2 g(n)$. 

В настоящей работе при $7<n\leq 20$ в качестве начальных приближений узлов интерполяции были использованы вершины симплексов максимальных объемов~в~$Q_n$. Эти симплексы были 
получены автором работы от научного руководителя.


\newpage
\section{Результаты} 

С помощью реализованной программы были получены численные верхние оценки минимальных норм проекторов для $n=1,\ldots,20$. Наиболее точные на момент выполнения работы оценки 
содержатся в книге~\cite{1}. Оценки для $n = 4$ и $n = 6$ были улучшены в работе \cite{10}.
Нам удалось найти более точные по сравнению с известными оценки при  $n = 5, 6, 10, 14, 18, 20$.

Наборы узлов на которых достигаются полученные оценки приводятся в~Приложении~2.
 
В Таблице~4.1 приводятся оценки, известные на настоящее время, и оценки, полученные при выполнении настоящей работы. Знаком  <<$*$>> отмечены оценки, которые удалось улучшить.

\begin{flushright}
	Таблица 4.1
\end{flushright}
\begin{center}
{\bf Сравнение известных оценок $\theta_n$ с оценками, \\ полученными в настоящей работе}
\end{center}
\begin{center}
	\begin{tabular}{|c|c|c|c|} \hline
		n& Известные оценки $\theta_n$ & Уточненные оценки $\theta_n$  & \\ \hline
		1 & $\theta_1 = 1$ & $\theta_1 = 1$ & \\ \hline
		2 & $\theta_2 = 1.89\ldots$ & $\theta_2 = 1.89\ldots$ & \\ \hline
		3 & $\theta_3 = 2$ & $\theta_3 = 2$ & \\ \hline
		4 & $2.2\ldots \leqslant\theta_n\leqslant2.3203\ldots$ & $2.2\ldots \leqslant\theta_n\leqslant2.3204\ldots$ & \\ \hline
		5 & $2.33\ldots \leqslant\theta_n\leqslant2.6\ldots$& $2.33\ldots \leqslant\theta_n\leqslant2.44880\ldots $ & $*$ \\\hline
		6 & $2.42\ldots \leqslant\theta_n\leqslant3$ & $2.42\ldots \leqslant\theta_n\leqslant2.60004\ldots$  &  $*$ \\ \hline
		7 & $\theta_7 = 2.5$& $\theta_7 = 2.5$  & \\ \hline
		8 & $2.5555\ldots\leqslant\theta_8\leqslant3.1428\dots$ &  $2.5555\ldots\leqslant\theta_8\leqslant3.1428\dots$  &  \\ \hline
		9 &$2.6\leqslant\theta_9\leqslant3.0000\ldots$ & $2.6\leqslant\theta_9\leqslant3.0000\ldots$ & \\ \hline
		10 &$2.6363\ldots\leqslant\theta_{10}\leqslant3.8000\dots$ &$2.6363\ldots\leqslant\theta_{10}\leqslant3.5186\dots$ &  $*$  \\ \hline
		11 &$2.6666\ldots\leqslant\theta_{11}\leqslant3.0000\dots$ &$2.6666\ldots\leqslant\theta_{11}\leqslant3.0000\dots$   & \\ \hline
		12 & $2.6923\ldots\leqslant\theta_{12}\leqslant3.4000\dots$& $2.6923\ldots\leqslant\theta_{12}\leqslant3.4000\dots$  & \\ \hline
		13 & $2.7142\ldots\leqslant\theta_{13}\leqslant3.7692\dots$ &$2.7142\ldots\leqslant\theta_{13}\leqslant3.7692\dots$  & \\ \hline
		14 &$2.7333\ldots\leqslant\theta_{14}\leqslant4.1999\dots$ & $2.7333\ldots\leqslant\theta_{14}\leqslant4.0156\dots$ &  $*$ \\ \hline
		15 & $2.75\ldots\leqslant\theta_{15}\leqslant3.5\dots$&  $2.75\ldots\leqslant\theta_{15}\leqslant3.5\dots$ & \\ \hline
		16 & $2.7647\ldots\leqslant\theta_{16}\leqslant4.2000\dots$ & $2.7647\ldots\leqslant\theta_{16}\leqslant4.2000\dots$ & \\ \hline
		17 &$2.7777\ldots\leqslant\theta_{17}\leqslant4.0882\dots$ &$2.7777\ldots\leqslant\theta_{17}\leqslant4.0882\dots$ &  \\ \hline
		18 & $2.7894\ldots\leqslant\theta_{18}\leqslant5.5882\dots$& $2.7894\ldots\leqslant\theta_{18}\leqslant5.14006\dots$ &  $*$\\ \hline
		19 &$2.8\leqslant\theta_{19}\leqslant4.0000\dots$ & $2.8\leqslant\theta_{19}\leqslant4.0000\dots$ & \\ \hline
		20 &$2.8095\ldots\leqslant\theta_{20}\leqslant4.7241\dots$ &$2.8095\ldots\leqslant\theta_{20}\leqslant4.68879\dots$ &  $*$ \\ \hline
	\end{tabular}
\end{center}

\newpage
\section{Заключение} 
%\addcontentsline{toc}{section}{Заключение}
В данной работе опробована возможность применения программных библиотек, таких как eigen, Dlib, boost для вычисления оценок минимальной нормы проектора. Использован алгоритм Бройдена-Флетчера-Голдфарба-Шанно (BFGS) для минимизации величины $\theta_n$ в $Q_n$. В результате быле найдены новые верхние оценки для величин  $\theta_5$, $\theta_6$,  $\theta_{10}$, $\theta_{14}$, $\theta_{18}$, $\theta_{20}$.  

\newpage
\begin{thebibliography}{1}
	\bibitem{1}
Невский М. В. Геометрические оценки в полиномиальной интерполяции. Ярославль: ЯрГУ, 2012.
	\bibitem{2}
King, D. E. (2009). Dlib-ml: A Machine Learning Toolkit. J. Mach. Learn. Res. 10 (Jul): 1755–1758
	\bibitem{3}
Eigen License. tuxfamily.org. Retrieved 16 Jan 2016
	\bibitem{4}	
Guennebaud, Gaël (2013). Eigen: A C++ linear algebra library. Eurographics/CGLibs.
	\bibitem{5}
Library Technical Report. JTC1/SC22/WG21 - The C++ Standards Committee. 2 July 2003. Retrieved 1 February 2012
	\bibitem{6}
Fletcher, Roger (1987), Practical methods of optimization (2nd ed.), New York: John Wiley \& Sons, ISBN 978-0-471-91547-8
	\bibitem{7}
Lewis, Adrian S.; Overton, Michael (2009), Nonsmooth optimization via BFGS
	\bibitem{8}
Nocedal, Jorge; Wright, Stephen J. (2006), Numerical Optimization (2nd ed.), Berlin, New York: Springer-Verlag, ISBN 978-0-387-30303-1
	\bibitem{9}
Byrd, Richard H.; Lu, Peihuang; Nocedal, Jorge; Zhu, Ciyou (1995), A Limited Memory Algorithm for Bound Constrained Optimization, SIAM Journal on Scientific Computing
	\bibitem{10}
	Кудрявцев, И. С., Озерова Е. А., Ухалов А.Ю. Новые оценки для норм минимальных проекторов, 2017


\end{thebibliography}
\newpage
\begin{flushright}
Приложение 1
\end{flushright}
\begin{center}
	{\bf Листинг программы }
\end{center}
\addcontentsline{toc}{section}{Приложение 1} 
\begin{listing}[1]{1}
#include <dlib/optimization.h>
#include <dlib/global_optimization.h>
#include <iostream>
#include <vector>
#include <iomanip>
#include <eigen3/Eigen/Dense>
#include <eigen3/Eigen/LU>
#include <boost/random.hpp>
#include <boost/random/normal_distribution.hpp>
#include <random>
#include <fstream>
#include <algorithm>
#include <string>
#include <iomanip>
#include <fstream>
#include <string>

using namespace dlib;
using namespace std;
using namespace Eigen;
using namespace std;
using namespace boost;

void LogMSG(const std::string &s)
{
	return;
	fstream out;
	out.open("/Users/artem/diplomlogs/diplomlogs.txt", ios::out);
	cout << s << endl;
	out.close();
}


int SYMPLEX_DIM;
typedef matrix<double,0,1> column_vector;

//Генератор случайных чисел
double sample(double dummy)
{
	using namespace std::chrono;
	std::default_random_engine engine(
	system_clock::to_time_t(system_clock::now()));
	std::uniform_real_distribution<> distr(0, 1);
	return distr(engine);
}

//Получение случайной матрицы
void getRandomMatrix( MatrixXd& a, int n)
{
	using namespace std::chrono;
	std::default_random_engine engine(
	system_clock::to_time_t( system_clock::now()) );
	std::uniform_real_distribution<> distr(0, 1);
	auto gen_number = [&engine, &distr] () { return distr(engine); };
	for( int i = 0; i <= n; ++i )
	{
		for( int j = 0; j < n; ++j )
		{
			a(i, j) = gen_number();
		}
	}
	
	for( int i = 0; i <= n; ++i )
	{
		a(i, n) = 1;
	}
	
}


//Вычисление нормы проектора
double getNorm( MatrixXd& a)
{
	LogMSG("getNorm/ start");
	int dim = SYMPLEX_DIM + 1;
	LogMSG("getNorm/ MatrixXd::inverse");
	MatrixXd invMatrix = a.inverse();
	//cout << a << endl;
	//cout << "detA = " << a.determinant() << endl;
	
	int pow2 = 1 << (dim-1);
	double norm = 0;
	LogMSG( "getNorm/ start search norm" );
	LogMSG( "getNorn/ number of cube vertex = " + std::to_string( pow2 ) );
	for(int i = 0; i < pow2; ++i)
	{
		std::vector<double> x(dim, 0);
		x[dim-1]=1;
		int ind = 0;
		int d = i;
		while( d )
		{
			x[ind] = d % 2;
			ind++;
			d /= 2;
		}
		double sum = 0;
		for( int k = 0; k < dim; ++k )
		{
			double cur_sum = 0;
			for( int j = 0; j < dim; ++j )
			{
				cur_sum += invMatrix(j,k)*x[j];
			}
			sum += std::abs( cur_sum );
		}
		
		norm = std::max( sum, norm );
	}
	LogMSG( "getNorm/ return norm" );
	return norm;
}

//Составить матрицу по вектору начального приближения
MatrixXd getMatixByVect( const column_vector& m )
{
	LogMSG("getMatixByVect/ start");
	MatrixXd A = MatrixXd::Zero( SYMPLEX_DIM+1,SYMPLEX_DIM+1 );
	for(int i = 0; i <= SYMPLEX_DIM; ++i)
	{
		for( int j = 0; j < SYMPLEX_DIM; ++j )
		{
			A(i, j) = m( SYMPLEX_DIM*i + j );
		}
		A(i, SYMPLEX_DIM) = 1;
	}
	return A;
}

//Составить матрицу по вектору
column_vector getVectorByMatrix( MatrixXd& A )
{
	column_vector vect( SYMPLEX_DIM*( SYMPLEX_DIM+1 ) );
	for( int i = 0; i <= SYMPLEX_DIM; ++i )
	{
		for( int j = 0; j < SYMPLEX_DIM; ++j )
		{
			vect(SYMPLEX_DIM*i + j) = A(i, j);
			
		}
	}
	return vect;
}

//Целевая функция для оптимизации
double func( const column_vector& m )
{
	LogMSG( "func/ start" );
	LogMSG( "func/ getMatrixByVect" );
	MatrixXd A =getMatixByVect( m );
	LogMSG( "func/ getNorm" );
	return getNorm( A );
}

//Метод для посчета минимальной нормы на
//случайно сгенерированном наборе симплексов
double MonteCarlo( int dimension )
{
	double minimalNorm = 1e9+7;
	for( int i = 0; i < 1e10; ++i )
	{
		MatrixXd a = MatrixXd::Zero( dimension+1,dimension+1 );
		getRandomMatrix( a, dimension );
		minimalNorm = std::min(minimalNorm, getNorm( a ));
	}
	return minimalNorm;
}

void solve( int n, column_vector& starting_point )
{
	try
	{
		//n-мерный случай
		SYMPLEX_DIM = n;
		cout << starting_point << endl;
		LogMSG( "main/ find_min_box_constrained" );
		cout << endl <<  find_min_box_constrained(
		bfgs_search_strategy(),
		objective_delta_stop_strategy( 1e-15 ).be_verbose(),
		func,
		derivative( func,1e-15 ),
		starting_point, 0, 1 );
		cout << "--------"<< endl;
		
		cout << "main/getMatixByVect" << endl;
		MatrixXd A1 = getMatixByVect( starting_point );
		cout << A1 << endl;
		cout << "main/getNorm" << endl;
		cout << "theta(" << n << ") = " << setprecision(11)
		<< getNorm(A1) << endl << "---------------\n";
	}
	catch(...){
		cout << "Процесс вычесления прерван" << endl;
	}
}

void test_solve()
{
	return;
}

void ReadDataFromConsole( int& n, column_vector& r_data )
{
	std::cout << "ReadDataFromConsole" << endl;
	std::cin >> n;
	r_data = column_vector( n*(n+1) );
	for( int i = 0; i <= n; ++i )
	{
		for( int j = 0; j < n; ++j )
		{
			double coord;
			std::cin >> coord;
			r_data(n*i + j) = coord;
		}
	}
	
	return;
}

void ReadDataFromFile( const std::string& file_name,
int &n,
column_vector& r_data )
{
	std::cout << "ReadDataFromFile" << endl;
	std::fstream input_file( file_name, ios_base::in );
	input_file >> n;
	r_data = column_vector( n*(n+1) );
	for( int i = 0; i <= n; ++i )
	{
		for( int j = 0; j < n; ++j )
		{
			double coord;
			input_file >> coord;
			r_data( n*i + j ) = coord;
		}
	}
	return;
}

void solve_with_random_start_point( int n )
{
	MatrixXd B = MatrixXd::Zero( n+1, n+1 );
	getRandomMatrix( B, n );
	column_vector starting_point = getVectorByMatrix( B );
	solve( n, starting_point );
}

int main() try
{
	
	int TYPE_INPUT = 0;
	// type input. Default input from console type = 0, from file type = 1;
	std::string NAME_INPUT_FILE;
	std::cin >> TYPE_INPUT;
	int n;
	column_vector starting_point;
	
	switch( TYPE_INPUT )
	{
		case 0:
		ReadDataFromConsole( n, starting_point );
		break;
		case 1:
		std::cin >> NAME_INPUT_FILE;
		ReadDataFromFile( NAME_INPUT_FILE, n, starting_point );
		break;
		default:
		std::cout << "Error input, check type input!";
		return 0;
		
	}
	
	//Примеры решения и стартовые приближения для n = 1..6
	
	solve( n, starting_point );
	
	column_vector starting_point2 = {0.45, 0.45,
		0.8, 0.69,
		0.3, 0.9};
	solve(2, starting_point2);
	column_vector starting_point3 = {0.547079, 0.0126172, 0.0867465,
		0.490059, 0.478761,  0.635784,
		0.93395, 0.514259, 0.193468,
		0.217008, 0.73415, 0.467616};
	solve(3, starting_point3);
	column_vector starting_point4 = {1, 0.292919, 0, 0,
		1, 1, 0.707131, 1,
		0.499911, 0, 1, 0.500076,
		0, 0.292961, 0, 1,
		0, 1, 0.70711, 0};
	solve(4, starting_point4);
	
	column_vector starting_point5 = {1, 1, 1, 0.282777, 0.662421,
		0, 0.282777, 0.662421, 0, 1,
		0.337578, 0, 1, 1, 0.282777,
		0, 1, 0.282777, 0.662421, 0,
		0.717222, 0.662421, 0, 1, 1,
		1, 0, 0, 0, 0};
	solve(5, starting_point5);
	column_vector starting_point6 = {1, 0, 1, 0.091, 0, 0.4999,
		0, 1, 1, 1, 0, 0,
		0.5, 1, 1, 0.091, 1, 1,
		0.9106, 0.0896, 0.0895, 1, 0.91053, 0.9106,
		0, 0, 0.5, 0.09115, 1, 0,
		0, 0.49999, 0, 0.09104, 0, 1,
		1, 1, 0, 0.09105, 0.5001, 0};
	solve(6, starting_point6);
	
	
}
catch (std::exception& e)
{
	cout << e.what() << endl;
}
catch(...)
{
	cout << "unknown exception"<<endl;
}
\end{listing}
\newpage
\begin{flushright}
Приложение 2
\end{flushright}
\begin{center}
	{\bf Наборы узлов интерполяции \\ на которых достигаются улучшенные оценки}
\end{center}
\addcontentsline{toc}{section}{Приложение 2} 
Далее для n, для которых улучшены верхние оценки минимальной нормы проектора $\theta_n$ приводятся симплексы, на которых достигается данная оценка, заданные координатами вершин 
(узлов интерполяции):
\begin{center}
{\bf\boldmath n=5, $\theta_{5}=$2.4488029336}\\
(1, 0.99999992367, 0.99999999995, 0.28277765266, 0.66242093126 ),
(1.9985588096e-07, 0.28277715142, 0.66242088676, 0, 0.99999999585),
(0.33757909432, 1.4555570776e-12,\\ 0.99999998881, 0.99999999384, 0.28277748041),
(0, 1, 0.28277711932, 0.66242011715,\\ 4.1747646609e-09),
(0.71722297192, 0.66242085342, 0, 0.99999999527, 0.99999982627),
(0.99999999766, 5.8842807186e-08, 7.8689584018e-08, 0, 1.231226984e-07) \\

{\bf\boldmath n=6, $\theta_{6}=$2.6000414905}\\
(0.99996732848, 4.404950382e-08, 0.99999803767, 0.091043705497, 1.5080287403e-05, 0.49999078839),
(1.5440523041e-05, 0.99998242532, 0.99998536235, 1, 4.2979109321e-06, 5.0154875895e-06),
(0.49997325566, 0.99999995279, 1, 0.091030248441, 0.99999971727, 0.99999485857),
(0.91054529517, 0.089496154482, 0.089462458908, 1, 0.91053597023, 0.91054722512),
(9.8896459314e-09, 5.0622915594e-06, 0.49998966987, 0.09106291499, 0.99999877544, 6.9374976785e-06),
(5.5000135932e-06, 0.50000006185, 1.5126932468e-07, 0.091050184422, 1.6459243787e-05, 0.99999751282),
(0.99998790427, 0.99999999534, 1.0226889906e-05, 0.091028761409, 0.50001817275, 3.3747301998e-07)

{\bf\boldmath n=10, $\theta_{10}=$3.5186849383}\\
(0.000343761, 0.283091, 0.00152227, 0, 0.993474, 0.97941, 0.962042, 0.989746, 0.00673603, 0.00435528),
(0.299096, 0.00174845, 1, 1, 0.00590979, 0.980404, 0.999164, 0.00352637, 0.000545441, 5.34969e-05),
(0.00125624, 0.999996, 0, 0.873951, 0.0999516, 0.900134, 0.903515, 0.139397, 0.998182, 0.99672),
(0, 0.939365, 0.900971, 0.997554, 0.999772, 0.0186347, 1.57508e-07, 0.971075, 0.136581, 0.143967),
(0.994957, 0.0107344, 0.0496297, 0.999201, 0.999864, 0.00234734, 0.932301, 0.118349, 0.00217507, 1),
(0.998827, 0.998919, 0.897163, 0.00384381, 0.0246995, 0.977905, 0.0455373, 0.938394, 0.00713544, 1),
(0.996961, 0.966364, 0.908957, 0.0394159, 0.945958, 0.0765708, 0.9765, 0.0194037, 1, 0.0054306),
(0.998516, 0.00311883, 0.0639222, 0.990097, 0.0869709, 0.915778, 0.00260304, 0.995842, 0.999934, 0.00539863),, 
(0.0345267, 0.00683825, 0.994703, 0.0919666, 0, 0, 1, 1, 0.99426, 0.99797),
(0.0416991, 0.00155994, 0.990757, 0.0810886, 0.989043, 0.997378, 7.50152e-05, 6.53832e-05, 0.982976, 1),
(0.000212241, 0.318944, 0.000215123, 0.000392483, 0.0358272, 0, 2.01195e-05, 0.0308336, 0.0691685, 0.0227924)

{\bf\boldmath n=14, $\theta_{14}=$4.0156094236}\\
(5.11975e-05, 0.0887167, 0.999972, 0.998949, 0.00174637, 0.000222371, 1, 0.00557882, 0.999992, 1.7163e-05, 0.000140542, 1, 3.02053e-05, 0.999274),
(0.119939, 0.118546, 0.998817, 0.00242529, 0.999721, 0.000253437, 0.000184339, 1, 0.999745, 0.00113355, 1, 0.000454303, 0.00135587, 0.999972),
(0.991502, 0.999999, 0.000333431, 0.999976, 0.0428634, 0.00192493, 0.033751, 1, 0.969658, 0.00212882, 0.999922, 0.969954, 0.954448, 0.0204009),
(0.00090201, 1, 0.0471678, 0.000420019, 3.94574e-05, 0.999958, 0.975597, 0.975096, 0.999999, 0.998514, 0.0235392, 0.0115776, 0.0367109, 0.999998),
(0.999866, 0.00070839, 0.989752, 0.961199, 0.0178236, 0.999985, 0.00012808, 0.963871, 0.0371852, 0.999959, 0.0162689, 0.00883974, 0.0510681, 0.993505),
(0.000109535, 0.000209244, 0.0077589, 0.999939, 1, 0.000255139, 0.989474, 0.999538, 1.2471e-05, 0.923057, 0.044588, 0.0592835, 0.992639, 0.947896),
(0.997899, 0.999684, 0.0456788, 0.974036, 0.999935, 0.954572, 0.999994, 0.0427239, 1.16546e-05, 0.00668326, 1, 0.0244723, 0.000320798, 0.968688)
(0.00395713, 0.00715372, 0.936955, 0.0494695, 0.0390959, 1, 0.995082, 0.987239, 0.000814255, 7.60102e-06, 0.999599, 0.998657, 0.912031, 0.0037748),
(1, 0.00139645, 0.982596, 0.0212883, 0.0334078, 0.053158, 0.968014, 0.000506459, 0.999927, 0.999997, 0.984547, 0.000470046, 1, 0.023132),
(0.000631627, 1, 0.999879, 0.000102598, 0.00201826, 0.0444134, 0.000420926, 0.00270765, 0.0425575, 0.999999, 0.977588, 0.976265, 0.999327, 0.999794),
(3.89041e-05, 1, 1, 0.980854, 0.97631, 0.999992, 0.0137901, 0.0255375, 0.973826, 0.0531728, 0.000364088, 0.00194644, 0.999845, 0.028632),
(0.999962, 0.000107765, 0.0328616, 2.48067e-06, 0.983354, 0.99966, 0.000144143, 0.0307265, 0.99075, 0.0432125, 3.73111e-05, 0.957377, 1, 0.998587),
(0.998942, 0.999302, 0.972642, 0.0199797, 1, 0.000174567, 0.980111, 0.999441, 0.00378433, 0.948393, 0.040194, 0.99999, 0.00144084, 1.6943e-05),
(0.00454873, 0.00233364, 0.00032794, 1, 0.999156, 0.926904, 0.0564063, 0.0400909, 0.953912, 0.999986, 0.999974, 0.999884, 2.75091e-05, 3.94854e-05),
(0.0871276, 0.000678347, 0, 1.34914e-06, 3.40856e-06, 0.000356143, 3.11508e-05, 0.000151653, 1.64089e-05, 0.000534129, 3.67674e-05, 0, 9.1822e-06, 2.34297e-06)

{\bf\boldmath n=18, $\theta_{18}=$5.140061}\\
(0, 6.7546e-09, 1, 0, 0.868399, 0.0497722, 0, 0.868399, 0.0497723, 1, 1, 1, 1, 0.0953266, 0.0953265, 0.0953264, 1, 1),
(2.28686e-08, 0, 1, 0.868399, 0, 0.0497723, 0.868399, 0, 0.0497722, 1, 1, 1, 0.0953265, 1, 0.0953266, 1, 0.0953265, 1),
(1, 1, 1, 0.0497722, 0.0497722, 1, 0.0497723, 0.0497722, 1, 0, 8.30747e-09, 8.30747e-09, 0.956133, 0.956133, 8.30748e-09, 0, 8.30748e-09, 0.956133),
(0, 0.868399, 0.0497722, 0, 6.7546e-09, 1, 0, 0.868399, 0.0497723, 0.0953265, 1, 1, 1, 1, 1, 1, 0.0953265, 0.0953266),
(0.868399, 0, 0.0497723, 2.28686e-08, 0, 1, 0.868399, 0, 0.0497722, 1, 0.0953265, 1, 1, 1, 1, 0.0953265, 1, 0.0953265),
(0.0497722, 0.0497722, 1, 1, 1, 1, 0.0497722, 0.0497722, 1, 0, 8.30749e-09, 0.956133, 0, 8.30748e-09, 8.30748e-09, 0.956133, 0.956133, 8.30748e-09),
(0, 0.868399, 0.0497723, 0, 0.868399, 0.0497722, 0, 6.7546e-09, 1, 1, 0.0953265, 0.0953265, 0.0953265, 1, 1, 1, 1, 1),
(0.868399, 0, 0.0497723, 0.868399, 0, 0.0497722, 2.28686e-08, 0, 1, 0.0953265, 1, 0.0953265, 1, 0.0953266, 1, 1, 1, 1),
(0.0497723, 0.0497723, 1, 0.0497723, 0.0497722, 1, 1, 1, 1, 0.956133, 0.956133, 8.30746e-09, 0, 8.30747e-09, 0.956133, 0, 8.30747e-09, 8.30747e-09),
(1, 1, 0, 0.0953265, 1, 0, 1, 0.0953265, 0.956133, 0.932512, 0.932512, 0.932512, 0.932512, 0, 0, 0.932512, 0, 0, 1),
(1, 1, 9.08861e-09, 1, 0.0953265, 9.08861e-09, 0.0953265, 1, 0.956133, 0.932512, 0.932512, 0.932512, 0, 0.932512, 6.56782e-09, 0, 0.932512, 6.56781e-09),
(1, 1, 9.08859e-09, 1, 1, 0.956133, 0.0953265, 0.0953265, 9.08859e-09, 0.932512, 0.932512, 0.932512, 0, 6.5678e-09, 0.932512, 0, 6.56782e-09, 0.932512),
(1, 0.0953265, 0.956133, 1, 1, 0, 0.0953264, 1, 0, 0.932512, 0, 0, 0.932512, 0.932512, 0.932512, 0.932512, 0, 0),
(0.0953265, 1, 0.956133, 1, 1, 9.0886e-09, 1, 0.0953265, 9.08859e-09, 0, 0.932512, 6.56781e-09, 0.932512, 0.932512, 0.932512, 0, 0.932512, 6.5678e-09),
(0.0953266, 0.0953265, 9.0886e-09, 1, 1, 9.08859e-09, 1, 1, 0.956133, 0, 6.56782e-09, 0.932512, 0.932512, 0.932512, 0.932512, 0, 6.56782e-09, 0.932512),
(0.0953265, 1, 0, 1, 0.0953265, 0.956133, 1, 1, 0, 0.932512, 0, 0, 0.932512, 0, 0, 0.932512, 0.932512, 0.932512),
(1, 0.0953264, 9.0886e-09, 0.0953265, 1, 0.956133, 1, 1, 9.08859e-09, 0, 0.932512, 6.56781e-09, 0, 0.932512, 6.56782e-09, 0.932512, 0.932512, 0.932512),
(1, 1, 0.956133, 0.0953265, 0.0953265, 9.08859e-09, 1, 1, 9.08859e-09, 0, 6.56781e-09, 0.932512, 0, 6.56781e-09, 0.932512, 0.932512, 0.932512, 0.932512),
(0.140881, 0.140881, 0, 0.140881, 0.140881, 0, 0.140881, 0.140881, 0, 2.3886e-08, 0, 0, 2.3886e-08, 0, 0, 2.3886e-08, 0, 0)

{\bf\boldmath n=20, $\theta_{20}=$4.688796}\\
(0.913198, 0, 0, 0, 0.00390006, 0.00390019, 0.00390012, 0.00390016, 0.947348, 0.947348, 0.947348, 0.947348, 0.947348, 0.947348, 0.947348, 0.947348, 0.947348, 0.947348, 0.947348, 0.947348),
(0, 0.913198, 0, 0, 0.947348, 0.947348, 0.947348, 0.947348, 0.0039002, 0.00390012, 0.00390015, 0.00390015, 0.947348, 0.947348, 0.947348, 0.947348, 0.947348, 0.947348, 0.947348, 0.947348),
(0, 0, 0.913198, 0, 0.947348, 0.947348, 0.947348, 0.947348, 0.947348, 0.947348, 0.947348, 0.947348, 0.0039001, 0.00390009, 0.00390011, 0.0039001, 0.947348, 0.947348, 0.947348, 0.947348),
(0, 0, 0, 0.913198, 0.947348, 0.947348, 0.947349, 0.947348, 0.947348, 0.947348, 0.947348, 0.947348, 0.947348, 0.947348, 0.947348, 0.947348, 0.00390015, 0.00390011, 0.00390015, 0.00390016),
(0.0039001, 0.947348, 0.947348, 0.947348, 0.885422, 0.0385256, 0.0385256, 0.0385257, 1, 1, 0, 0, 1, 1, 0, 1.99812e-08, 0, 1, 1, 2.03086e-08),
(0.00390011, 0.947348, 0.947348, 0.947348, 0.0385256, 0.885422, 0.0385256, 0.0385256, 1, 0, 1, 1.69286e-08, 0, 0, 1, 1, 1, 1, 1.82243e-08, 0),
(0.0039001, 0.947348, 0.947348, 0.947348, 0.0385256, 0.0385256, 0.885422, 0.0385257, 1.69286e-08, 1, 0, 1, 2.08925e-08, 1, 0, 1, 1, 1.58687e-08, 0, 1),
(0.00390025, 0.947348, 0.947348, 0.947348, 0.0385257, 0.0385257, 0.0385256, 0.885422, 0, 0, 1, 1, 1, 0, 1, 0, 1.78263e-08, 0, 1, 1),
(0.947348, 0.0039001, 0.947348, 0.947348, 0, 0, 1, 1, 0.885422, 0.0385257, 0.0385256, 0.0385256, 1, 1.78636e-08, 0, 1, 1, 1.59377e-08, 1, 1.91314e-08),
(0.947348, 0.00390016, 0.947348, 0.947348, 1, 1, 0, 0, 0.0385256, 0.885422, 0.0385257, 0.0385256, 1, 0, 1, 0, 1, 0, 1.92302e-08, 1),
(0.947348, 0.0039001, 0.947348, 0.947348, 1, 0, 2.05184e-08, 1, 0.0385257, 0.0385257, 0.885422, 0.0385256, 2.37411e-08, 1, 0, 1, 0, 1, 0, 1),
(0.947348, 0.00390012, 0.947348, 0.947348, 2.05184e-08, 1, 1, 0, 0.0385256, 0.0385257, 0.0385256, 0.885422, 0, 1, 1, 0, 1.80565e-08, 1, 1, 0),
(0.947348, 0.947348, 0.00390009, 0.947348, 1, 1.78636e-08, 1, 0, 0, 0, 1, 1, 0.885422, 0.0385256, 0.0385256, 0.0385257, 1, 1, 1.91476e-08, 0),
(0.947348, 0.947348, 0.0039001, 0.947348, 2.37411e-08, 1, 0, 1, 0, 1, 1, 2.08925e-08, 0.0385256, 0.885422, 0.0385256, 0.0385257, 1, 0, 1, 0),
(0.947348, 0.947348, 0.00390011, 0.947348, 0, 0, 1, 1, 1, 1, 0, 0, 0.0385256, 0.0385256, 0.885422, 0.0385257, 0, 1, 0, 1),
(0.947348, 0.947348, 0.00390014, 0.947348, 1, 1, 0, 0, 1, 0, 1.99812e-08, 1, 0.0385257, 0.0385256, 0.0385257, 0.885422, 1.87415e-08, 0, 1, 1),
(0.947348, 0.947348, 0.947348, 0.00390021, 1, 1.81351e-08, 0, 1, 1, 0, 1.9251e-08, 1, 0, 1, 1, 0, 0.885422, 0.0385255, 0.0385256, 0.0385256),
(0.947348, 0.947348, 0.947348, 0.00390019, 0, 1, 1.69101e-08, 1, 1.58538e-08, 1, 0, 1, 1, 1.6657e-08, 0, 1, 0.0385256, 0.885422, 0.0385256, 0.0385257),
(0.947348, 0.947348, 0.947348, 0.00390025, 1, 1.6934e-08, 1, 0, 0, 1, 1, 1.68607e-08, 0, 0, 1, 1, 0.0385257, 0.0385257, 0.885422, 0.0385256),
(0.947348, 0.947348, 0.947348, 0.00390014, 0, 1, 1, 2.03644e-08, 1, 2.03224e-08, 1, 0, 1, 1, 0, 2.13658e-08, 0.0385256, 0.0385257, 0.0385256, 0.885422),
(0, 0, 0, 0, 0, 0, 3.38164e-08, 3.34541e-08, 3.1704e-08, 3.33852e-08, 0, 0, 3.91276e-08, 3.33102e-08, 0, 3.50991e-08, 0, 2.97189e-08, 0, 3.56742e-08)
\end{center}

\end{document}