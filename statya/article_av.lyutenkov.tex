%Сборник трудов молодых ученых 2018 г. Вып. 18 (математика, информатика)
\documentclass[14pt,a4paper]{extbook}
%\usepackage{pscyr} 
\usepackage{mathtext}
\usepackage[T2A]{fontenc}
\usepackage[utf8x]{inputenc}
\usepackage[russian]{babel}
\usepackage{graphicx}
\usepackage{multicol}
\usepackage{amsfonts,amssymb,amscd,amsmath,amsthm,wrapfig}
\usepackage[left=2.5cm, right=2.5cm, top=2.7cm, bottom=2cm]{geometry}
\usepackage{epsf}
\usepackage{indentfirst}
\usepackage{my}
\usepackage{floatflt}
\usepackage{cite}


\hbadness5000
\vbadness3000
\sloppy
\clubpenalty=10000
\widowpenalty=10000
\floatingpenalty=20000


\renewcommand{\theequation}{\arabic{equation}}
\renewcommand{\thefigure}{\arabic{figure}}
\renewcommand{\thetable}{\arabic{table}}
\renewcommand{\contentsname}{Содержание}


%всякие макросы авторов, очень желательно не увлекаться


\def\labelenumi{\theenumi.}      % чтобы после номера шла точка;

\begin{document}

\begin{flushleft}
\copyright~Лютенков, А.\,В., 2018
\end{flushleft}

{\sl УДК~517.518.238+519.6}\\ % Номер УДК
\medskip

\begin{center}
{\bf А.\,В.~Лютенков}\\[0.2cm] %% Автор
{\Large Минимальные нормы интерполяционных проекторов}% Название
%%Если работа поддержана каким-либо фондом
%\footnote[1]{Работа выполнена при финансовой поддержке ...}
\end{center}

\addcontentsline{toc}{section}{\textit{Лютенков, А.\,В.} Минимальные нормы интерполяционных проекторов} \markboth{А.\,В.~Лютенков} {Сборник научных трудов молодых ученых \ldots\;.
Вып.~18 (2018)}

%% Аннотация
\begin{quotation}
\small В работе рассматривается задача о построении минимального проектора (проектора, имеющего минимальную норму) при интерполяции непрерывной на кубе функции с помощью полиномов $n$ переменных степени не выше единицы. 
Автором разработана компьютрная программа для численной минимизации функции 
многих переменных. С помощью этой программы удалось уточнить верхние границы 
минимальных норм проекторов при $n=5,6,10,14,18,20$.
\end{quotation}


\subsection*{1. Задача линейной интерполяции на n-мерном кубе}  % Или {1. Заголовок}, т.е. нумерация "руками"

%\subsubsection*{Подзаголовок}
Положим $Q_ n:= [0..1]^n$, где $n \in \mathbb{R}^n$, $Q_ n$ --- $n$-мерный куб, множество вершин куба будем обозначать как $ver(Q_n)$.  $\Pi_1(\mathbb{R}^n)$ --- совокупность многочленов $n$ переменных степени $\leqslant 1$. Пусть S --- невырожденный сиплекс в $\mathbb{R}^n$, вершины симплекса зададаются, как $x^{(j)} = (x_1^{(j)},...,x_n^{(j)}),  j = 1,\ldots,n $. Рассмотрим матрицу
$$A := {\begin{pmatrix}
	x_1^{(1)}& \dots & x_n^{(1)}& 1\\
	\vdots & \ddots & \vdots & \vdots \\
	x_1^{(n+1)}& \dots & x_n^{(n+1)}&  1
	\end{pmatrix}}. \eqno (1.1)$$
Будем говорить, что набор точек $x^{(j)}$ --- допустим для интерполяции многочленами из $\Pi_1(\mathbb{R}^n)$. Это условие эквивалентно тому, что
матрица A является невырожденной.  
\newline
$\Delta := det(A)$, определитель, который получается  из  $\Delta$ заменой j-й строки на строку $(x_1, \dots, x_n, 1)$. Многочленый $\lambda_j(x) := \Delta_j(x)/\Delta$ из $\Pi_1(\mathbb{R}^n)$ называются базисными многочленами Лагранжа симплекса S и обладают свойством  $\lambda_j(x^{k}) = \delta^k_j $, где $\delta^k_j $ --- символ Кронекера. $\lambda_j = l_{1j}x_1 + \dots + l_{nj}x_n + l_{n+1j}$, коэффициентны $l_{ij}$ составляют столбцы матрицы $$A^{-1} = {\begin{pmatrix}
	\dots & l_{1,j}&\dots\\
	\vdots & \vdots & \vdots \\
	\dots& l_{n,j} & \dots\\
	\dots& l_{n+1,j} & \dots\\
	\end{pmatrix}}.\eqno (1.2)$$
\newline
Так как $\lambda_j(x^{k}) = \delta^k_j $ любой многочлен $p \in \Pi_1(\mathbb{R}^n)$ удовлетворяет равенству 
$$p(x) = \sum\limits_{j = 1}^{n+1} p(x^{(j)})\lambda_j(x). \eqno (1.3)$$
\newline
Так как $det(A) \neq 0 $, то для любой  $f \in C(Q_n)$, где $C(Q_n)$ --- совокупность $f : Q_n \rightarrow \mathbb{R}$ найдется единственный многочлен $p \in \Pi_1(\mathbb{R}^n	)$ удовлетворяющий условиям:
$$p(x^{(j)}) = f(x^{(j)}). \eqno (1.4)$$
\subsection*{1.1 Интерполяционный проектор}
Введем в рассмотрение оператор $P : C(Q_n)  \rightarrow \Pi_1(\mathbb{R}^n)$, который далее будем называть интеполяционным проектором. Интерполяционный проектор по системе узлов $x^{(j)}$ определяется с помощью равенств:
$$Pf(x^{(j)}) = f_j := f(x^{(j)}),  j = 1,\dots, n+1. \eqno (1.1.1)$$

Из этих равенств следует, что данный оператор является линейным и справедлив следующий аналог интерполяционной формулы Лагранжа:
$$Pf(x^{(j)}) = p(x) = \sum\limits_{j=1}^{n+1} f_j\lambda_j(x). \eqno(1.1.2)$$

\subsection*{1.2 Норма интерполяционного проектора. Минимальная норма проектора }
Обозначим через $||P||$ норму проектора  $P$ как оператора из $C(Q_n)$
в $C(Q_n)$. Эта величина зависит от от выбора узлов интерполяции  $x^{(j)}$. 

%\begin{lemma}
\smallskip
\noindent {\bf Лемма 1.} {\sl
	Для любого интерполяционного проектора $P : C(Q_n)\rightarrow \Pi_1(\mathbb{R}^n)$ и симплекса $S$ с вершинами в его узлах имеет место равенство 
	$$||P|| = \max\limits_{x \in ver(Q_n)} \sum\limits_{j = 1}^{n+1} |\lambda_j(x)|\eqno (1.2.1)$$}
\smallskip
%\end{lemma}
Доказательство этого утверждения можно найти в монографии [1].
\newline
Обозначим через $\theta_n$ минимальну норму проектора, при условии, что все узлы принадлежат кубу $Q_n$:
$$\theta_n := \min\limits_{x^{(j)} \in Q_n}||P||\eqno (1.2.2)$$
Интерполяционный проектор $P^*$ c нормой $||P^*|| = \theta_n$ назовем минимальным. 

Главной задачей настоящей работы является уточнение оценок для миниммальной нормы проектора при некоторых значениях $n$. 

Отметим, что минимальная норма проектора достигается на границе куба $Q_n$, т.е. в том случае, когда все вершины невырожденного симплекса принадлежат границе $Q_n$. Доказательство данного факта приводится в книге М.В. Невского [1].

В монографии [1] приводятся следующие общие оценки:
$$\frac{1}{4}\sqrt{n}< \theta_n <3\sqrt{n},$$
$$3-\frac{4}{n+1}\leqslant\theta_n.$$


\section*{2. Компьютерная программа для оценивания~$\theta_n$} 

Была реализована компьютерна программа для уточнения оценок $\theta_n$. Задача об отыскании минимальной нормы проектора сводится к задаче отыскания минимума функции многих переменных. 
Норма проектора вычисляется по формуле(1.2.1), зная это, зададим целевую функцию для минимизации. 
$$F(A) = \max\limits_{x \in ver(Q_n)} \sum\limits_{j = 1}^{n+1} |\lambda_j(x)|\eqno (2.1.1)$$
Где А --- марица, которая имеет вид (1.1).

Так как задача минимизации функции (2.1.1) является трудоемкой, целесообразным является применение сторонних программных библиотек, поставляющих решения задач линейной алгебры, численных методов и медодов оптимизации, существующих на рынке свободно распространяемого програмного обеспечения.

Программа реализована на языке программировани C++ с использованием библиотек Dlib, Boost, Eigen, которые предоставляют необходимый функционал для оптимального решения поставленной задачи.

\subsection*{Реализация программы}
Программа реализована в виде консольного приложения. Приложение зависит от выше указанных программных библиотек, также использует страндартную библиотеку шаблонов STL языка C++. 

Так как для вычисления нормы проектора необходим перебор по вершинам куба, то асимптотика алгоритма отыскания $\theta_n$ будет порядка $O(C\cdot n^2\cdot 2^n)$, что требует оптимального выполнения некоторых операций, для обеспечения наилучшего времени работы программы. Для оптимальной работы с матрицами и операций над ними (напимер вычисление обратной матрицы) используется библиотека Eigen.

Для оптимизации целевой функции (2.1.1) используются возможности библиотеки Dlib. Dlib предоставляет метод минимизации нелинейной функции многих переменных внутри куба $Q_n$, с возможностью использования различных стратегий поиска минимума функции. Для решения текущей задачи стратегиями поиска были выбраны алгоритмы: BFGS и L-BFGS.


\section*{3. Результаты} 

С помощью реализованной программы были получены численные верхние оценки минимальных норм проекторов для $n=1,\ldots,20$. Наиболее точные на момент выполнения работы оценки 
содержатся в книге [1]. Оценки для $n = 4$ и $n = 6$ были улучшены в работе [2].
Нам удалось найти более точные по сравнению с известными оценки при  $n = 5, 6, 10, 14, 18, 20$.

В Таблице~3.1 приводятся оценки, известные на настоящее время, и оценки, полученные при выполнении настоящей работы. Знаком  <<$*$>> отмечены оценки, которые удалось улучшить.

\newpage
\begin{flushright}
	Таблица 3.1
\end{flushright}
\begin{center}
	{\bf Сравнение известных оценок $\theta_n$ с оценками, \\ полученными в настоящей работе}
\end{center}
\begin{center}
	\begin{tabular}{|c|c|c|c|} \hline
		n& Известные оценки $\theta_n$ & Уточненные оценки $\theta_n$  & \\ \hline
		1 & $\theta_1 = 1$ & $\theta_1 = 1$ & \\ \hline
		2 & $\theta_2 = 1.89\ldots$ & $\theta_2 = 1.89\ldots$ & \\ \hline
		3 & $\theta_3 = 2$ & $\theta_3 = 2$ & \\ \hline
		4 & $2.2\ldots \leqslant\theta_n\leqslant2.3203\ldots$ & $2.2\ldots \leqslant\theta_n\leqslant2.3204\ldots$ & \\ \hline
		5 & $2.33\ldots \leqslant\theta_n\leqslant2.6\ldots$& $2.33\ldots \leqslant\theta_n\leqslant2.44880\ldots $ & $*$ \\\hline
		6 & $2.42\ldots \leqslant\theta_n\leqslant3$ & $2.42\ldots \leqslant\theta_n\leqslant2.60004\ldots$  &  $*$ \\ \hline
		7 & $\theta_7 = 2.5$& $\theta_7 = 2.5$  & \\ \hline
		8 & $2.5555\ldots\leqslant\theta_8\leqslant3.1428\dots$ &  $2.5555\ldots\leqslant\theta_8\leqslant3.1428\dots$  &  \\ \hline
		9 &$2.6\leqslant\theta_9\leqslant3.0000\ldots$ & $2.6\leqslant\theta_9\leqslant3.0000\ldots$ & \\ \hline
		10 &$2.6363\ldots\leqslant\theta_{10}\leqslant3.8000\dots$ &$2.6363\ldots\leqslant\theta_{10}\leqslant3.5186\dots$ &  $*$  \\ \hline
		11 &$2.6666\ldots\leqslant\theta_{11}\leqslant3.0000\dots$ &$2.6666\ldots\leqslant\theta_{11}\leqslant3.0000\dots$   & \\ \hline
		12 & $2.6923\ldots\leqslant\theta_{12}\leqslant3.4000\dots$& $2.6923\ldots\leqslant\theta_{12}\leqslant3.4000\dots$  & \\ \hline
		13 & $2.7142\ldots\leqslant\theta_{13}\leqslant3.7692\dots$ &$2.7142\ldots\leqslant\theta_{13}\leqslant3.7692\dots$  & \\ \hline
		14 &$2.7333\ldots\leqslant\theta_{14}\leqslant4.1999\dots$ & $2.7333\ldots\leqslant\theta_{14}\leqslant4.0156\dots$ &  $*$ \\ \hline
		15 & $2.75\ldots\leqslant\theta_{15}\leqslant3.5\dots$&  $2.75\ldots\leqslant\theta_{15}\leqslant3.5\dots$ & \\ \hline
		16 & $2.7647\ldots\leqslant\theta_{16}\leqslant4.2000\dots$ & $2.7647\ldots\leqslant\theta_{16}\leqslant4.2000\dots$ & \\ \hline
		17 &$2.7777\ldots\leqslant\theta_{17}\leqslant4.0882\dots$ &$2.7777\ldots\leqslant\theta_{17}\leqslant4.0882\dots$ &  \\ \hline
		18 & $2.7894\ldots\leqslant\theta_{18}\leqslant5.5882\dots$& $2.7894\ldots\leqslant\theta_{18}\leqslant5.14006\dots$ &  $*$\\ \hline
		19 &$2.8\leqslant\theta_{19}\leqslant4.0000\dots$ & $2.8\leqslant\theta_{19}\leqslant4.0000\dots$ & \\ \hline
		20 &$2.8095\ldots\leqslant\theta_{20}\leqslant4.7241\dots$ &$2.8095\ldots\leqslant\theta_{20}\leqslant4.68879\dots$ &  $*$ \\ \hline
	\end{tabular}
\end{center}


{\large\bf Литература}

\begin{enumerate}

\item 
{\it Невский М. В.} Геометрические оценки в полиномиальной
интерполяции. Ярославль: ЯрГУ, 2012.

\item 
{\it Кудрявцев, И. С., Озерова Е. А., Ухалов А.Ю.}
 Новые оценки для норм минимальных проекторов // Современные проблемы математики и информатики: сборник научных трудов молодых ученых, аспирантов и студентов. Вып. 17. / Яросл. гос. ун-т им.~П.\,Г.~Демидова. Ярославль: ЯрГУ, 2017. С.~74--81.
\end{enumerate}


\begin{flushright}
{\it Ярославский государственный университет им. П.\,Г. Демидова}
\end{flushright}

\end{document}