% Шапка документа - не менять

\documentclass[12pt, a4paper]{extarticle}

\usepackage{amsfonts}

%\usepackage[T2A]{fontenc}

\usepackage[cp1251]{inputenc}

\usepackage{mathtext}

\usepackage{amsmath, amsfonts, amssymb}

\usepackage[russian]{babel}

\usepackage[body={17.5cm, 23.5cm},left=2cm, top=2cm]{geometry}

\sloppy

\binoppenalty=10000

\relpenalty=10000
% Макросы для учёных степеней

\newcommand{\kfmn}{{\mdseries канд.~физ.-мат.~наук}}

\newcommand{\ken}{{\mdseries канд.~эконом.~наук}}

\newcommand{\kpn}{{\mdseries канд.~пед.~наук}}

\newcommand{\khn}{{\mdseries канд.~хим.~наук}}

\newcommand{\dpn}{{\mdseries доктор~пед.~наук}}

\newcommand{\dfmn}{{\mdseries доктор~физ.-мат.~наук}}

%ученые звания

\newcommand{\doc}{{\mdseries доцент}}

\newcommand{\prof}{{\mdseries профессор}}

%Заголовок аннотации

\newcommand{\atitle}[1]{\begin{center}{\Large #1\par}\end{center}}

%автор

\newcommand{\auth}[2]{\noindent{\bf #1}, #2 курс\addcontentsline{toc}{subsection}{#1}\par}

%научный руководитель

\newcommand{\swise}[1]{\noindent Научный руководитель: {\bfseries #1\par}}

\newcommand{\coswise}[1]{\noindent Соруководитель: #1\par}

% Текст

\begin{document}

% Название работы, тема курсовой, диплома.

\atitle{Алгоритм восстановления невыпуклой триангулированной поверхности по облаку точек}

% Фамилия Имя Отчество автора и курс, на котором выполнена работа.

\auth{Лютенков Артем Вадимович}{3}

% учёная степень, ученое звание Фамилия и инициалы научного руководителя

% Учёные степени (расшифровки есть выше):

% \kfmn, \ken, \kpn, \khn, \dpn, \dfmn

% если у научного руководителя иная степень, то оформить ее словами - новый макрос заводить не нужно

% Учёные звания (расшифровки есть выше):

% \doc, \prof

% Если у научного руководителя отсутствует учёная степень и/или звание, то не пишется ничего

\swise{Преображениска М. М.}

В данной курсовой работе рассматривается решение задачи восстановления поверхности по облаку точек.  
Допустим, нам дано множество точек в  $\mathbb{R}^d$, где $d = 2$ или $d=3$, этот набор точек мы будем называть облаком точек. Требуется получить представление о форме поверхности задаваемой этими точками. Форма --- вполне неопределенное понятие и есть,  много возможных интерпретаций определения данного обьекта. Самое простое решение --- это построить выпуклую оболочку (рис. 1). Но очевидно, что данный вариант дает нам неполное представление о форме поверхности задаваемой данным множеством точек. На рисунке 1 мы можем видеть, что точки окружают пустую область круглой формы.  Хотелось бы учитывать это при получении поверхности и получить изображение, более отражающее действительность (рис. 2). Такую возможность нам предоставляет использование $\alpha$-shapes (определение будет дано в данной работе) и алгорим Эдельсбруннера, используещий данное понятие (будет описан далее). 

\medskip