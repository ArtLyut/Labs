% Шапка документа - не менять

\documentclass[12pt]{article} % Класса article хватит для курсовой и диплома 
\usepackage[utf8]{inputenc}
\usepackage[english,russian]{babel}
\usepackage{amsfonts}

%\usepackage[T2A]{fontenc}

\usepackage[cp1251]{inputenc}

\usepackage{mathtext}

\usepackage{amsmath, amsfonts, amssymb}


\usepackage[body={17.5cm, 23.5cm},left=2cm, top=2cm]{geometry}

\sloppy

\binoppenalty=10000

\relpenalty=10000

% Определения некоторых макросов - не менять

% Макросы для учёных степеней

\newcommand{\kfmn}{{\mdseries канд.~физ.-мат.~наук}}

\newcommand{\ken}{{\mdseries канд.~эконом.~наук}}

\newcommand{\kpn}{{\mdseries канд.~пед.~наук}}

\newcommand{\khn}{{\mdseries канд.~хим.~наук}}

\newcommand{\dpn}{{\mdseries доктор~пед.~наук}}

\newcommand{\dfmn}{{\mdseries доктор~физ.-мат.~наук}}

%ученые звания

\newcommand{\doc}{{\mdseries доцент}}

\newcommand{\prof}{{\mdseries профессор}}

%Заголовок аннотации

\newcommand{\atitle}[1]{\begin{center}{\Large #1\par}\end{center}}

%автор

\newcommand{\auth}[2]{\noindent{\bf #1}, #2 курс\addcontentsline{toc}{subsection}{#1}\par}

%научный руководитель

\newcommand{\swise}[1]{\noindent Научный руководитель: {\bfseries #1\par}}

\newcommand{\coswise}[1]{\noindent Соруководитель: #1\par}

% Текст

\begin{document}

% Название работы, тема курсовой, диплома.

\atitle{Аннотация}
%\atitle{Курсовая работа на тему: Алгоритм восстановления невыпуклой триангулированной поверхности по облаку точек.}
Курсовая работа на тему: Алгоритм восстановления невыпуклой триангулированной поверхности по облаку точек.

% Фамилия Имя Отчество автора и курс, на котором выполнена работа.
Работу выполнил: 
\auth{Лютенков Артем Вадимович}{3}

% учёная степень, ученое звание Фамилия и инициалы научного руководителя

% Учёные степени (расшифровки есть выше):

% \kfmn, \ken, \kpn, \khn, \dpn, \dfmn

% если у научного руководителя иная степень, то оформить ее словами - новый макрос заводить не нужно

% Учёные звания (расшифровки есть выше):

% \doc, \prof

% Если у научного руководителя отсутствует учёная степень и/или звание, то не пишется ничего

    \swise{Преображенская М.М.}

\medskip

В данной курсовой работе рассматривается задача построения поверхности по заданному множеству точек S в $\mathbb{R}^2$ или $\mathbb{R}^3$. Используются такие понятия, как триангуляция Делоне, клетка Вороного, симплициальный помплекс,  $\alpha$-комплекс, $\alpha$-shapes; алгоритм Эдельсбруннера построения поверхности по облаку точек. 
\newline
Алгоритм Эдельсбруннера:
\newline
1. Вычислить триангуляцию Делоне (DT(S)), зная, что граница $\alpha$-shape содержится в ней.
\newline
2. Затем мы определяем $C_\alpha$(S) путем проверки всех симплексов  $\Delta_T$ в DT (S): если
$\sigma_T$-шар вокруг $\mu_T$ пуст и $\sigma_T$ $ < \alpha$ (это альфа-тест), мы
принимаем  $\Delta_T$, как член $C_\alpha$(S), вместе со всеми его гранями
\newline
3. Все d-симплексы $C_\alpha$(S) составляют внутренность $S_\alpha$. Все симплексы на
границе $\partial C_\alpha$ составляют границу $\alpha$-shape $\partial S_\alpha$.
\newline
 Также в работе рассмотрен вопрос об интеграции функционала, предоставляемого пакетами (alphashapes, alphahull, geometry) языка программирования R, в программу для построения 3D-моделей и стереометрических чертежей 3dSchoolEdit. Для интеграции в 3dSchoolEdit будет использоваться RCaller.
RCaller --- библиотека для вызова R кода из Java. RCaller преобразует структуры данных в R код, отправляет их внешнему R процессу, возвращает сгенерированные результаты XML формате. Структура XML анализируется и возвращает значения доступные непосредственно в Java.


\medskip


\end{document}