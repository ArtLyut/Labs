\documentclass[12pt]{article} %  ласса article хватит дл€ курсовой и диплома 
\usepackage[utf8]{inputenc}
\usepackage[english,russian]{babel}
\usepackage[T2A,T1]{fontenc}
\usepackage{amsmath} % ѕакет специальных символов
\usepackage{amsfonts} 
\usepackage{amssymb} % Ётих пакетов хватает дл€ набора большинства спецсимволов 
\usepackage{graphicx} % √рафика
\usepackage{hyperref} % Ќавигаци€ по ссылкам в документе % ѕродвинутое цитирование
\usepackage{cite}
\usepackage{color} %% это дл€ отображени€ цвета в коде
\usepackage{listings} %% собственно, это и есть пакет listings
\usepackage{bm}
\usepackage{caption}
\DeclareCaptionFont{white}{\color{white}} %% это сделает текст заголовка белым
%% код ниже нарисует серую рамочку вокруг заголовка кода.
\DeclareCaptionFormat{listing}{\colorbox{black}{\parbox{\textwidth}{#1#2#3}}}
\captionsetup[lstlisting]{format=listing,labelfont=white,textfont=white}
\usepackage[a4paper, top=25mm, left=30mm, right=10mm, bottom=25mm]{geometry} % ѕол€

\title{¬ычисление геометрических характеристик симплекса}
\author{Yury.Dubov}
\date{May 2017}

\begin{document}
	\begin{titlepage}
		 \begin{center}
			\large
			\textbf{ћ»ЌќЅ–Ќј” » –ќ——»»}
			
			
			\vspace{0.5cm}
			
		\textbf{'едеральное государственное бюджетное образовательное'}
		
		\textbf{учреждение высшего образовани€}
		
		\textbf{Ђярославский государственный университет им. ѕ.√. ƒемидоваї}
			\vspace{0.25cm}
			
			
			 афедра математического анализа
			\vfill
			
			 урсова€ работа \\
			\textbf{¬ычисление геометрических характеристик симплекса} \\
			(—пециальность 01.03.02 ѕрикладна€ математика и информатика) 
			\vfill
			
			
			\bigskip
			
			
		\end{center}

\begin{flushright}
Ќаучный руководитель\\
\underline{\hspace{5cm}}\\
 \small(степень, звание)\\
 \underline{\hspace{2.5cm}} \hspace{0.1cm}  \underline{\hspace{2.5cm}}\\
 \small(подпись,‘»ќ)\\
 Ђ\underline{\hspace{0.5cm}}ї \underline{\hspace{1.5cm}} 20 \underline{\hspace{0.2cm}} г.\\
 —тудент группы ѕћ»-31Ѕќ\\
 \underline{\hspace{2.5cm}} \hspace{0.1cm}  \underline{\hspace{2.5cm}}\\
 \small(подпись,‘»ќ)\\
 Ђ\underline{\hspace{0.5cm}}ї \underline{\hspace{1.5cm}} 20 \underline{\hspace{0.2cm}} г.\\
\end{flushright}
		\begin{center}
			ярославль, 2017 г.
		\end{center}
\end{titlepage}

\tableofcontents
\newpage
\section{¬ведение}

\subparagraph{}¬ данной курсовой работе, с помощью программы написанной на €зыке $C\#$, будут вычислены некоторые геометрические характеристики симплексов. 
\subparagraph{}ѕусть $n \in N$, элемент $x \in R^{n}$ будем записывать в виде $x = (x_1, \dots, x_n)$. „ерез $e_1, \dots , e_n$ обозначаетс€ канонический базис $R^{n}$; cчитаем $e := (1,\dots, 1)$. ƒл€ $x \in R^{n}$ через $\|x\|$ ниже обозначаетс€ обычна€ евклидова норма $x$:
	
	\begin{equation*}
	\|x\|:= ({\sum_{i=1}^n x_i^{2}})^{1/2}.
	\end{equation*}

	\subparagraph{}ѕусть $C$ Ч выпуклое тело в $R^n$, т. е. компактное выпуклое подмножество $R^n$ с непустой внутренностью. 
	\subparagraph{}„ерез $\sigma C$ обозначим результат гомотетии $C$ относительно центра т€жести с коэффициентом $\sigma$. Ѕудем говорить, что n-мерный симплекс описан вокруг выпуклого тела $C$, если кажда€ (n-1)-мерна€ грань этого симплекса содержит точку $C$. ѕримем по определению, что выпуклый многогранник вписан в $C$, если люба€ его вершина принадлежит границе $C$.
	
	\newtheorem{Def}{ќпределение}
	\begin{Def} —имплекс(размерности n) --- это выпукла€ оболочка $n+1$ точки аффинного пространства, которые предполагаютс€ аффинно независимыми (то есть не лежат в подпространстве размерности $n-1$). Ёти точки называютс€ вершинами симплекса.
	\end{Def}
	
	\subparagraph{} ѕусть $S$ Ч невырожденный симплекс в $R^n$: ќбозначим вершины $S$ через $x^{(j)}:=\{x_1^{(j)},\dots,x_n^{(j)}\}$, $j=1,\dots,n+1$. ћатрица $A$ €вл€етс€ невырожденной:
	
	\begin{equation}\label{f1}
		A :=
		\begin{pmatrix}
			x_1^{(1)} & \dots & x_n^{(1)} & 1 \\
			x_1^{(2)} & \dots & x_n^{(2)} & 1 \\
			\vdots  & \dots  & \dots  & \vdots \\
			x_1^{(n+1)} & \dots & x_n^{(n+1)} & 1
		\end{pmatrix}
	\end{equation}
	
	
	\subparagraph{} ќбозначим через $\Delta_j(x)$ опpеделитель, который получаетс€ из $\Delta$ заменой $j$-й строки на строку $(x_1,\dots, x_n, 1)$. –ассмотрим многочлены $\lambda_j(x) := \Delta_j(x)/\Delta$ из $\Pi_1(R^n)$, они обладают свойством $\lambda_j(x(k)) = \delta_j^k$ (здесь $\delta_j^k$ Ч символ  ронекера).  оэффициенты $\lambda_j$ составл€ют $j$-й столбец обратной к $A$ матрице $A^{-1}$: ¬ дальнейшем считаем $A^{-1} = (l_{ij})$, иначе говор€,
	
	\begin{equation}\label{f2}
		\lambda_j(x) = l_{1,j}x_1 + \dots + l_{n,j}x_{n} + l_{n+1,j}.
	\end{equation}
	
	\newtheorem{Def2}[Def]{ќпределение}
	\begin{Def2}ƒл€ выпуслого тела $—$ обозначим через $d_i(C)$ максимальную длину отрезка, содержащегос€ в $C$ и параллельного оси $x_i$ и будем называть $i$-м осевым диаметром $C$.
	\end{Def2}
	
	\newtheorem{Def5}[Def]{ќпределение}
	\begin{Def5}
		ѕусть $S$ --- невырожденный симплекс, $C$ --- выпуклое тело в $R^n$. ¬ведЄм в рассмотрение величину
	\end{Def5}
	
	\begin{equation}\label{f9}
	\xi(C, S):=\min \{\sigma \geq 1:\ C \subset \sigma S\}.
	\end{equation}
	
	\subparagraph{} ѕоложим $\xi (S):=\xi (Q_n;S)$. ќчевидно, $\xi(C; S)=1$ тогда и только тогда, когда $C \subset S$.
	
	\newtheorem{Def6}[Def]{ќпределение}
	\begin{Def6}
		ѕусть $C=Q_n$, S --- невырожденный $n$-мерный симплекс. ¬ведЄм в рассмотрение характеристику
		\begin{equation}\label{f11}
			\xi_n = \min \{\xi(S):\ S \subset Q_n\}.
		\end{equation}
	\end{Def6}	 
	
	»так, целью данной работы €вл€етс€ вычисление таких характеристик симплекса, как $\xi (S)$ и $\xi_n$. ќ том, как они вычисл€ютс€, будет более подробно изложенно в следующих главах работы.
	
	\newpage
	
	
	
	
	
	
	\section{¬еличина $\xi (C; S)$}   	
	\newtheorem{Th}{“еорема}
	\begin{Th}ѕусть $C \not \subset S$ и $1 \leq j \leq n$. ѕредположим, что j-€ $(n-1)$-мерна€ грань симплекса
	$\xi(C, S)S$ (параллельна€ грани $S$ c уравнением $\lambda_j(x)=0$) содержит точку $C$. “огда
		
		\begin{equation}\label{f10}
		\xi(C, S) = (n+1) \max_{x \in C} (-\lambda_j(x)) + 1.
		\end{equation}
	\end{Th}
	
	\newtheorem{Th2}{—ледствие}
	\begin{Th}ѕусть $S$ --- невырожденный симплекс, $C$ --- выпуклое тело в $R^n$. ѕредположим $C \not \subset S$. ≈сли симплекс $\xi(S)S$ описан вокруг $C$, то
		
		\begin{equation}\label{theorem2}
		    \max_{x \in C} (-\lambda_1(x))=\dots=\max_{x \in C} (-\lambda_{n+1}(x))
		\end{equation}
	\end{Th}
	
	\newtheorem{Th3}[Th]{“еорема}
	\begin{Th3}ѕусть $S$ --- невырожденный симплекс, $C$ --- выпуклое тело в $R^n$ и $C \not \subset S$, тогда 
		
		\begin{equation}\label{theorem3}
		\xi(C, S) = (n+1) \max_{1\leqslant k \leqslant n+1} \max_{x \in C} (-\lambda_k(x)) + 1.
		\end{equation}
	\end{Th3}
	
	ќтдельно остановимс€ на случае $C = Q_n$. ћногочлен из $\Pi_1(R^n)$ принимает минимальное и максимальное значени€ на $Q_n$ в вершинах куба. ¬ св€зи с этим в соотношени€х насто€щего пункта величина $max_C(-\lambda_j)$ при $C=Q_n$ может быть заменена на равную величину $\max_{ver(Q_n)}(-\lambda_j)$. ¬ частности, равенство (\ref{theorem3}) принимает вид
	
    \begin{equation}\label{ksiVer}
		\xi(C, S) = (n+1) \max_{1\leqslant k \leqslant n+1} \max_{x \in ver(Q_n)} (-\lambda_k(x)) + 1.
	\end{equation}
	а условие (\ref{theorem2}) сводитс€ к соотношению
	
	\begin{equation}\label{f10}
	    \max_{x \in ver(Q_n)} (-\lambda_1(x))=\dots=\max_{x \in ver(Q_n)} (-\lambda_{n+1}(x))
	\end{equation}

    Ќапомню, что
    $$\xi_n = \min \{\xi(S):\ S \subset Q_n\}.$$
	√де $C=Q_n$, S --- невырожденный $n$-мерный симплекс. 
	
	\subparagraph{}—тоит упом€нуть, что были найдены точные значени€ $\xi_n$ дл€ случаев $n = 1, 2, 3$ и соответствующие им симплексы.
	
	\subparagraph{}ѕри $n = 1\ \xi_1 = 1$. ƒл€ любого проектора существует 1-вершина $Q_1$ относительно соответствующего симплекса (который в этой ситуации €вл€етс€ отрезком).
	
	\subparagraph{}ѕри $n = 2\ \xi_2 = \dfrac{3\sqrt{5}}{5} + 1$, причем симплекс, соответствующий $\xi_2$, обладает интересным свойством: его одномерные грани(стороны треугольника) отсекают от квадрата треугольники равных площадей.
	
	\subparagraph{}ѕри $n = 3\ \xi_3 = 3$, причЄм существует только два симплекса, которым соответствует $\xi_3$, и что интересно, их двумерные грани(плоскости, ограничивающие симплекс) отсекают от куба фигуры равных объЄмов.
	\newpage







\section{–абота программы}
\subparagraph{}ќписание работы программы стоит разделить на две части. ѕерва€ будет рассказывать о нахождении $\xi(S)$, по заданным координатам вершин соответствующего симплекса. ¬тора€ опишет процесс нахождени€ $\xi_n$ по заданной размерности $n$.
јлгоритм первой части рассмотрим на примере. »так, n-мерный симплекс задаЄтс€ через n+1 точку n-мерного пространства. «анесЄм данные в матрицу, как показано в (\ref{f1}), в результате получаем  (\ref{pr1})
\begin{equation}\label{pr1}
		A :=
		\begin{pmatrix}
			0 & 0 & 1 \\
			0 & 1 & 1 \\
			1 & 0.5 & 1
		\end{pmatrix}
\end{equation}
ƒалее дл€ (\ref{pr1}) находим обратную матрицу.
\begin{equation}\label{pr2}
		A^{-1} :=
		\begin{pmatrix}
			-0.5 & -0.5 & 1 \\
			-1 & 1 & 0 \\
			1 & 0 & 0
		\end{pmatrix}
\end{equation}
—оставим базисные многочлены Ћагранжа:
$$\lambda_1(x)=-0.5x_1-x_2+1$$
$$\lambda_2(x)=-0.5x_1+x_2$$
$$\lambda_3(x)=1$$
¬оспользуемс€ формулой (\ref{ksiVer}), где $C=Q_n$. ќчевидно, что $-\lambda_j(x)$ достигает своего максимума в точке $x=(x_1,x_2,\dots,x_n)$, где
\begin{equation*}
x_i = 
 \begin{cases}
   1, &\text{если $-l_{i,j}>0$}\\
   0, &\text{если $-l_{i,j}\le0$}
 \end{cases}
\end{equation*}
ƒл€ нашего примера имеем:
$$\max(-\lambda_1(x))=0.5 \text{при} x_1=x_2=1$$
$$\max(-\lambda_2(x))=0.5 \text{при} x_1=1, x_2=0$$
$$\max(-\lambda_3(x))=0 \text{при} x_1=0$$
‘ормула (\ref{ksiVer}) даЄт $\xi(S)=3*0.5+1=2.5$
ѕроцесс выполнени€ программы практически ничем не отличаетс€ от описанного выше алгоритма.

\subparagraph{}¬тора€ часть программы занимаетс€ приближЄнным вычислением $\xi_n$, путЄм условно-полного перебора всех n-мерных симплексов в $Q_n$. ”словность заключаетс€ в том, что на самом деле перебираютс€ не все возможные симплексы, коих несчЄтное множество, а с некоторым малым шагом. т.е. кажда€ координата каждой вершины симплекса мен€етс€ в диапазоне от 0 до 1 с шагом в 0.001, и дл€ каждого такого симплекса находитс€ $\xi(S)$ по вышеуказанной процедуре. ѕосле чего по формуле (\ref{f11}) выбираетс€ минимальный $\xi(S)$. “ак, например, $\xi_2 = 2,3418\dots$, а $\xi_3 = 3$. 



\newpage
\section{«аключение}
¬ данной работе поднимаетс€ така€ актуальна€ проблема в математике, как изучение геометрических характеристик симплекса - $\xi(S)$ и $\xi_n$. Ѕыли повторно вычислены $\xi_2$ и $\xi_3$. ¬ычисление $\xi_n$ при большем $n$ затруднительно, из-за необходимости перебора огромного количества вариантов, хот€ это и возможно. „то касаетс€ вычислени€ $\xi(S)$, то даже при достаточно больших $n$, врем€ выполнени€ программы будет мало.

\newpage
% даЄм указание на включение данного место в оглавление как секции (\section)
\addcontentsline{toc}{section}{—писок используемой литературы}
\begin{thebibliography}{}
	\bibitem{link1} ћ.¬.Ќевский - √еометрические оценки в полиномиальной интерпол€ции. ярославль, 2012г.
	\bibitem{link1} “роелсен Ё. - язык программировани€ C$\#$ 5.0 и платформа .NET 4.5. 2015г.
\end{thebibliography}


\newpage
\section{ѕриложение}
\begin{verbatim}
using System;
using System.Collections.Generic;
using System.Linq;
using System.Text;
using System.Threading.Tasks;

namespace Matrix
{
    class Simplex
    {
        int size;//n характеристика
        public Matrix matrix;
        public Matrix array;//та же сама€ матрица, но без столбца единиц
        public int Size { get { return size; } }

        public Simplex(Simplex simplex)
        {
            size = simplex.Size;
            this.matrix = new Matrix(size + 1, size + 1);            
            this.array = new Matrix(size + 1, size);
            
            for (int i = 0; i < size + 1; i++)
            {
                for (int j = 0; j < size; j++)
                {
                    this.matrix.array[i, j] = simplex.matrix.array[i, j];
                    this.array.array[i, j] = simplex. matrix.array[i, j];
                }
                this.matrix.array[i, size] = simplex.matrix.array[i, size];
            }
        }

        public Simplex(Matrix matrix)//конструктор
        {
            if(matrix.Row!=matrix.Column+1)
            {
                throw new Exception("ћатрица не подходит");
            }
            this.matrix = new Matrix(matrix.Row, matrix.Column + 1);
            this.array = new Matrix(matrix.Row, matrix.Column);
            size = matrix.Column;
            for(int i = 0; i<size + 1; i++)
            {
                for(int j = 0; j<size; j++)
                {
                    this.matrix.array[i, j] = matrix.array[i, j];
                    this.array.array[i, j] = matrix.array[i, j];
                }
                this.matrix.array[i, size] = 1;
            }
        }

        public Decimal LambdaMax()
        {
            Matrix inverseMatrix = new Matrix(this.matrix.Inverse());
            Decimal[] lambda_i = new Decimal[size+1];
            for (int i = 0; i <= size; i++)
                lambda_i[i] = 0;
            //находим максимальное значение дл€ -Ћ€мбда_i 
            for (int j=0; j<=size;j++)
            {
                for (int i=0; i<size; i++)
                {
                    if(inverseMatrix.array[i,j]<0)
                    {
                        lambda_i[j] -= inverseMatrix.array[i, j];
                    }
                }
                lambda_i[j] -= inverseMatrix.array[size, j];
            }
            return lambda_i.Max();
        }


        public Decimal Ksi()
        {
            decimal lambda = LambdaMax();
            return (size + 1) * lambda + 1;
        }


        //обновл€ет матрицу симплекса на новую
        public void RefreshMatrix(Matrix matr) 
        {
            if (matr.Row != matr.Column + 1 && matr.Row != this.matrix.Row)
            {
                throw new Exception("ћатрица не подходит");
            }
            for (int i = 0; i < size + 1; i++)
            {
                for (int j = 0; j < size; j++)
                {
                    this.matrix.array[i, j] = matr.array[i, j];
                    this.array.array[i, j] = matr.array[i, j];
                }
                this.matrix.array[i, size] = 1;
            }
        }
        
        //обновл€ет матрицу симплекса на новую 
        public void RefreshMatrix(Decimal[,] Array)
        {
            Matrix newMatrix =
                new Matrix(this.matrix.Row, this.matrix.Column-1, Array);
            RefreshMatrix(newMatrix);
        }


    }
}

\end{verbatim}

\begin{verbatim}
using System;
using System.Collections.Generic;
using System.Linq;
using System.Text;
using System.Threading.Tasks;
using System.IO;

namespace Matrix
{
    class Simplex_n
    {
        int size;//n характеристика(размер пространства)
        public int Size
        {
            get { return size; }
            set { size = value; }
        }

        public Simplex_n(int Size)
        {
            size = Size;
        }

        //провер€ет на наличие 0 и 1
        public bool CheckHave1Or0(Decimal[] point)
        {
            for (int i = 0; i < size; i++)
                if (point[i] == 0 || point[i] == 1.000m)
                    return true;
            return false;
        }
        
        //по точке возвращает следующую
        public Decimal[] NewPoint(Decimal[] point)
        {
            Decimal[] newPoint = new Decimal[size];
            for (int i = 0; i < size; i++)
                newPoint[i] = point[i];
            int j = size - 1;

            while (j >= 0)
            {
                if (newPoint[j] <= 0.999m)
                {
                    newPoint[j] += 0.001m;
                    return newPoint;
                }
                /*уменьшаем j до тех пор,
                пока нельз€ будет прибавить 0,001*/
                while (j >= 0 && newPoint[j] > 0.999m) 
                {
                    j--;
                }
                /*можем увеличить j-ую координату на 0,001*/
                for (int k = j + 1; k < size; k++)
                    newPoint[k] = 0;
            }
            return point;
            //возвращаем саму точку, если не можем перейти к следующей.
            //т.е. вектор уже из всех единиц
        }

        //получени€ следующей матрицы, зна€ предыдущую
        public Decimal[,] NextMatrix(Decimal[,] Matrix, int Row)
        {
            Matrix matr = new Matrix(Row, Row - 1, Matrix);
            int k = Row;//индекс на единицу больше последней строки
            Decimal[] tmp; //текуща€ строка
            Decimal[] nextTmp;
            do
            {
                k--;
                tmp = matr.GetRow(k);
                nextTmp = NewPoint(tmp);
                while(!CheckHave1Or0(nextTmp))
                    nextTmp = NewPoint(nextTmp);
            }
            //ищем строку, которую можно преобразовать в следующую
            while (k > 0 && Comparison(tmp, nextTmp));

            //если строчку можно преобразовать
            if (!Comparison(tmp, nextTmp))
            {
                matr.SetRow(k, nextTmp);
                for (int i = k + 1; i < Row; i++)
                    for (int j = 0; j < Row - 1; j++)
                        matr.array[i, j] = 0m;
            }
            return matr.array;
        }

        //проверка на равенство двух массивов
        public bool Comparison(Decimal[] m1, Decimal[] m2)
        {
            if (m1.Length != m2.Length)
                return false;
            int i = 0;
            while (i < m1.Length)
            {
                if (m1[i] != m2[i])
                    return false;
                i++;
            }
            return true;
        }
        
        //проверка на равенство двух матриц
        public bool Comparison(Matrix m1, Matrix m2)
        {
            if (m1.Row != m2.Row || m1.Column != m2.Column)
                return false;
            for (int i = 0; i < m1.Row; i++)
                for (int j = 0; j < m1.Column; j++)
                    if (m1.Array[i, j] != m2.Array[i, j])
                        return false;
            return true;
        }

        //проверка на то, что достигли конца перебора матриц
        public bool CheckOnFinish(Simplex simplex)
        {

            for (int i = 0; i < simplex.array.Row; i++)
                for (int j = 0; j < simplex.array.Column; j++)
                    if (simplex.array.array[i, j] != 1.000m)
                        return false;

            return true;
        }
        
        //пропускаем те симплексы, у которых не можем посчитать кси
        public void Skip(ref Simplex sim)
        {
            while (sim.matrix.Determinant() == 0)
            {
                sim.RefreshMatrix(NextMatrix(sim.matrix.Array, sim.matrix.Row));
                if (CheckOnFinish(sim))
                    break;
            }
        }

        public Decimal Ksi_n()
        {
            Decimal[,] array = new Decimal[size + 1, size];
            for(int i = 0; i < size+1; i++)
                for(int j = 0; j < size; j++)
                    array[i, j] = 0m;
            Matrix matrix = new Matrix(size + 1, size, array);
            Simplex sim = new Simplex(matrix);

            Skip(ref sim);
            Decimal minKsi = sim.Ksi();
            Decimal ksi;

            sim.RefreshMatrix(NextMatrix(sim.matrix.Array, sim.matrix.Row));
            Skip(ref sim);

            do
            {
                ksi = sim.Ksi();
                if (ksi < minKsi)
                    minKsi = ksi;

                sim.RefreshMatrix(NextMatrix(sim.matrix.Array, sim.matrix.Row));
                Skip(ref simplex);
            }
            while (!CheckOnFinish(sim));
            return minKsi;
        }

    }
}

\end{verbatim}

\begin{verbatim}
using System;
using System.Collections.Generic;
using System.Linq;
using System.Text;
using System.Threading.Tasks;

namespace Matrix
{
    public class Matrix
    {
        static Random rnd = new Random();

        public Decimal[,] array;
        int row, column;

        public Decimal[,] Array
        {
            get { return array; }
            set { array = value; }
        }

        public int Row { get { return row; } }

        public int Column { get { return column; } }

        public Matrix(int row, int column, Decimal [,] array)
        {
            this.row = row;
            this.column = column;
            this.array = new Decimal[row, column];
            for(int i = 0; i<row; i++)
                for(int j = 0; j<column; j++)
                {
                    this.array[i, j] = array[i, j];
                }
        }
        public Matrix(int row, int column)
        {
            this.row = row;
            this.column = column;
            array = new Decimal[row, column];
        }
        public Matrix(Matrix matrix)
        {
            this.column = matrix.Column;
            this.row = matrix.Row;
            this.array = new Decimal[row, column];
            for (int i = 0; i < row; i++)
                for (int j = 0; j < column; j++)
                {
                    this.array[i, j] = matrix.array[i, j];
                }
        }

        public void Random()
        {
            for (int i = 0; i < row; i++)
            {
                for (int j = 0; j < column; j++)
                {
                    array[i, j] = rnd.Next(10);
                }
            }
        }

        public void Random(int min, int max)
        {
            for (int i = 0; i < row; i++)
            {
                for (int j = 0; j < column; j++)
                {
                    array[i, j] = rnd.Next(min, max);
                }
            }
        }

        public Matrix Transpose()
        {
            Matrix m = new Matrix(column, row);

            for (int i = 0; i < row; i++)
            {
                for (int j = 0; j < column; j++)
                {
                    m.array[j, i] = array[i, j];
                }
            }

            return m;
        }

        public void TransposeMyself()
        {
            array = Transpose().array;
        }

        public Matrix Inverse()
        {
            Decimal det = Determinant();
            if (det == 0)
            {
                throw new Exception("ћатрица вырождена");
            }

            Matrix m = new Matrix(row, column);

            for (int i = 0; i < row; i++)
            {
                for (int j = 0; j < column; j++)
                {
                    m.array[i, j] = Cofactor(array, i, j) / det;
                }
            }

            return m.Transpose();
        }

        public Decimal Determinant()
        {
            if (column != row)
            {
                throw new Exception("–асчет определител€ невозможен");
            }
            return Determinant(array);
        }

        private Decimal Determinant(Decimal[,] array)
        {
            int n = (int)Math.Sqrt(array.Length);

            if (n == 1)
            {
                return array[0, 0];
            }

            Decimal det = 0;

            for (int k = 0; k < n; k++)
            {
                det += array[0, k] * Cofactor(array, 0, k);
            }

            return det;
        }

        private Decimal Cofactor(Decimal[,] array, int row, int column)
        {
            return Convert.ToDecimal(Math.Pow(-1, column + row)) * 
                                Determinant(Minor(array, row, column));
        }

        private Decimal[,] Minor(Decimal[,] array, int row, int column)
        {
            int n = (int)Math.Sqrt(array.Length);
            Decimal[,] minor = new Decimal[n - 1, n - 1];

            int _i = 0;
            for (int i = 0; i < n; i++)
            {
                if (i == row)
                {
                    continue;
                }
                int _j = 0;
                for (int j = 0; j < n; j++)
                {
                    if (j == column)
                    {
                        continue;
                    }
                    minor[_i, _j] = array[i, j];
                    _j++;
                }
                _i++;
            }
            return minor;
        }

        public Decimal[] GetRow(int IndexRow)
        {
            Decimal[] resault = new Decimal[column];
            for(int i = 0; i < column; i++)
            {
                resault[i] = array[IndexRow, i];
            }
            return resault;
        }

        public bool SetRow(int IndexRow, Decimal[] Row)
        {
            if (column != Row.Length || IndexRow < 0 || IndexRow > row - 1)
                return false;
            for(int i = 0; i < column; i++)
            {
                array[IndexRow, i] = Row[i];
            }
            return true;
        }

        public static Matrix operator +(Matrix m1, Matrix m2)
        {
            if (m1.row != m2.row || m1.column != m2.column)
            {
                throw new Exception("—ложение невозможно");
            }

            Matrix m = new Matrix(m1.row, m1.column);

            for (int i = 0; i < m1.row; i++)
            {
                for (int j = 0; j < m1.column; j++)
                {
                    m.array[i, j] = m1.array[i, j] + m2.array[i, j];
                }
            }

            return m;
        }

        public static Matrix operator -(Matrix m1, Matrix m2)
        {
            if (m1.row != m2.row || m1.column != m2.column)
            {
                throw new Exception("¬ычитание невозможно");
            }

            Matrix m = new Matrix(m1.row, m1.column);

            for (int i = 0; i < m1.row; i++)
            {
                for (int j = 0; j < m1.column; j++)
                {
                    m.array[i, j] = m1.array[i, j] - m2.array[i, j];
                }
            }

            return m;
        }

        public static Matrix operator *(Matrix m1, Matrix m2)
        {
            if (m1.column != m2.row)
            {
                throw new Exception("”множение невозможно");
            }

            Matrix m = new Matrix(m1.row, m2.column);

            for (int i = 0; i < m1.row; i++)
            {
                for (int j = 0; j < m2.column; j++)
                {
                    decimal sum = 0;

                    for (int k = 0; k < m1.column; k++)
                    {
                        sum += m1.array[i, k] * m2.array[k, j];
                    }

                    m.array[i, j] = sum;
                }
            }

            return m;
        }

        public override string ToString()
        {
            string str = "";

            for (int i = 0; i < row; i++)
            {
                for (int j = 0; j < column; j++)
                {
                    str += array[i, j] + "\t";
                }
                str += "\n";
            }

            return str;
        }
    }
}

\end{verbatim}
\end{document}